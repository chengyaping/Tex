%%
%%% >>> Published papers
%%
\begin{publications}

{\bf e$^+$/e$^-$ Discrimination in Liquid Scintillator and Its Usage to Suppress $^8$He/$^9$Li Backgrounds }
[to be publish]


{\bf Measurement of the Reactor Antineutrino Flux and Spectrum at Daya Bay}

F. P. An {\it et al.}, [Daya Bay Collaboration]

Phys. Rev. Lett. {\bf 116}, 061801 (2016)


{\bf Spectral Measurement of Electron Antineutrino Oscillation Amplitude and Frequency at Daya Bay}

F. P. An {\it et al.}, [Daya Bay Collaboration]

Phys. Rev. Lett. {\bf 112}, 061801 (2014) 
\end{publications}

%%
%%% >>> Resume if necessary
%%
%\begin{resume}
%
%\paragraph{CASthesis~作者基本情况}
%吴凌云,男,福建省屏南县人,1975 年出生,中国科学院数学与系统科学研究院博士研究生。
%
%\paragraph{联系方式}
%通讯地址:北京市~2734 信箱,中科院数学与系统科学研究院应用数学所
%
%邮编:100080
%
%E-mail: aloft@ctex.org
%
%\paragraph{ucasthesis~作者基本情况}
%莫晃锐,男,湖南省湘潭县人,1989 年出生,中国科学院力学研究所硕士研究生。
%
%\paragraph{联系方式}
%通讯地址:北京市北四环西路15号中国科学院力学研究所
%
%邮编:100190
%
%E-mail: mohuangrui@gmail.com
%
%\end{resume}

%%
%%% >>> Acknowledgements
%%
\begin{thanks}
值此论文完成之际,谨在此向多年来给予我关心和帮助的老师、同学、朋友
和家人表示衷心的感谢!

我的导师王贻芳研究员在论文的选题、课题的研究、论文的撰写等方面给予了关键的指导。
工作的每一步都得到了王老师耐心的指导,在此对王老师致以衷心的感谢!王老师渊博的学
识,高瞻远瞩的眼光,认真的工作态度对我今后的学习和工作都将产生深远巨大的影响。


感谢我的课题指导老师曹俊研究员。论文的完成得益于曹老师日常的谆谆教诲。曹老师百科
全书式的物理知识,开阔的思路,敏捷的思维以及严谨的工作作风,都深深影响了我。在此
表示诚挚的感谢!

特别感谢高能所中微子组的何苗,温良剑,占亮,李玉峰,Soeren Jetter,曹国富,郭万磊,王志民、路浩奇、
徐吉磊、安丰鹏,钱森 ,Macro Grassi对我工作的
大力支持和帮助。他们和我一起讨论问题,让我感受到合作在科学研究中的重要性。
另外也要特别感谢C103的全体成员,赵洁,胡维,阳马生,赵庆旺,张玄同,丁雪峰等,和以你们一起学习工作非常的荣幸。


最后,我要感谢我的亲人们,我的一切都来自于他们对我的深情培育。快乐生
活、努力工作是我报效他们的唯一方式。

\end{thanks}
