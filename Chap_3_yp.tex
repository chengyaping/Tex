
\chapter{光电倍增管的刻度及测试}
\label{chap:chap3}
在中微子实验中,闪烁体或水通常被用于建造大型中微子探测器,需要使用数以万计的光电倍增管~PMT~探测闪烁光或切伦科夫光。当光打到置于真空中的金属或者半导体上时,电子就会从其表面发射到真空中,被称为光电效应。%pmt_handbook
光电倍增管就是基于光电效应的一种光电转换器件。由于PMT是一种无损探测器件,并且具有响应速度快,灵敏度高等优点,除了高能物理外,还被广泛运用到医学设备以及工业设备中。

\numberwithin{equation}{section}

\section{大亚湾中心探测器PMT增益刻度}
\subsection{Rolling Gain 刻度方法介绍}
大亚湾实验中用的PMT增益刻度方法有两种,一种是通过调节驱动~LED~的电压,使~LED~发出单光子,单光子打到
~PMT~上,从而研究~PMT~单光子响应的~LED~ 增益刻度方法。这种方法依赖与~LED~的精细调节,而且需要牺牲物理
取数时间,专门做这些增益刻度~run~。另一种使用来自非光源的单光电子进行刻度的~Rolling Gain~ 刻度方法。使
用的单光电子主要来自~PMT~的暗噪声以及随机触发数据。这种方法不需要~LED~,而且在物理取数的同时,可以得到
刻度数据,提供了~PMT~暗噪声率的在线监测方法,对于在线实时处理数据也有好处。做~Rolling Gain~ 刻度的第一
步是刻度数据的挑选。下面是被挑选光子需要满足的一些条件:
\begin{itemize}
\item 距离前一个触发的时间间隔大于20$mu$s
\item 该Hit的peak cycle 在【4,6】之间
\item 每个PMT通道只有一个Hit
\item ADC工作在精细量程区域
\item 电子学基线恢复良好,要求$ \left \| preAdc-avePed \right \| <20 $
\item 需要去除40MHz的来自电子学时钟的噪声。方法是:(Tdc-15)\%16>=4
\item TDC位于噪声窗口,及大于1070。另要求噪声窗口内的~hit multiplicity~ 小于3
\end{itemize}
Hit挑选完之后,需要使用~PMT~单光子相应的~Model~去对数据进行~Fit~。 这里使用的模型来自
\citep{bellamy1994absolute}。在该模型中,光电倍增管被划分为两个相对独立的功能模块。一是模拟光阴极的光子电
子转换功能。另一是描述打拿级系统对原初光电子的倍增放大过程。光子转换成光电子的效率由PMT的量子效率决
 定。考虑到PMT 的收集效率,假设由第一打拿级搜集到的平均光电子数目为~$\mu$~,则实际观测到的光电子数目
 ~$n$~与~$\mu$~满足泊松分布。$P(n;mu)=\frac{\mu^{n}e^{-\mu}}{n!}$。 电子的倍增放大过程,由于
 ~PMT~自身的电荷分辨,可以用一个高斯分布描述。实际的PMT响应的描述可以表示成如下模式~\ref{eq:model}~。
 \begin{eqnarray} \label{eq:model}
 S_{real}(x) &=& \int S_{ideal}(x^{\prime})B(x-x^{\prime})dx^{\prime}\\
 S_{ideal}(x) &=& P(n;\mu)\otimes G_n(x) \nonumber \\
 &=& \sum_{n=0}{\infty} \frac{{\mu}^n e^{-\mu}}{n!}\frac{1}{\sigma_1\sqrt{2n\pi}}exp(-\frac{{(x-nQ_1)}^2}{2n{\sigma_1}^2}) \\
B(x)&=&\frac{1-w}{\sigma_0\sqrt{2\pi}}exp(-\frac{x^2}{2\sigma_0^2})+w\theta (x) \alpha exp(-\alpha x ) \end{eqnarray} 
  这个模型被用到~Rolling Gain~的~Fit~当中。一个好的~Fit~结果需要满足一下条件:
~Root Fit~的状态为成功。~chi2 ~除以~ndf~需要小于~10~。 刻度出来的增益常数在[6,80]的范围内,并且在边界
内。~Fit ~结果的误差需要合理,通常要求小于10。~gain~ 的sigma要求小于gain 值得1/3. gain sigma的绝对值在
[1,30] 的范围内。~Fit~的示例如图~\ref{fig:rg}~。
\begin{figure}[!htbp]
  \centering
   \includegraphics[width=\MyFactor\textwidth]{Img/chap3/rg}
    \caption{~Rolling Gain~ 刻度示例}
  \label{fig:rg}
\end{figure}
得到刻度常数以后,需要更新刻度参数表。其基本流程如下~\ref{fig:db}~
首先在本底拷贝一个offline database的副本。往本底副本数据库里插入刻度参数。对本底副本里插入的刻度参数做有效性检验。联系负责人,更新官方离线数据库。
\begin{figure}[!htbp]
  \centering
  \begin{subfigure}[b]{\MySubFactor\textwidth}
    \includegraphics[width=\MyFactor\textwidth*5/3]{D:/Thesis/ucasthesis-master/Img/chap3/db1}
    \caption{建立本地副本}
    \label{fig:db_1}
  \end{subfigure}%
  \quad\quad\quad\quad%add desired spacing
  \begin{subfigure}[b]{\MySubFactor\textwidth}
    \includegraphics[width=\MyFactor\textwidth*5/3]{D:/Thesis/ucasthesis-master/Img/chap3/db2}
    \caption{检验上传}
    \label{fig:db_2}
  \end{subfigure}
  \caption{刻度参数数据库操作}
  \label{fig:db}
\end{figure}

除了提供增益刻度参数外,~Rolling Gain~还提供了暗噪声率除强制触发外的另外一种计算方法。暗噪声窗口为~TDC~从1070 到1170,共100个TDC,1个~TDC~转换为ns为1.5625ns。然后使用暗噪声窗口内的总~hit~数除以暗噪声窗口长度,即可得到暗噪声率。该方法与强制触发方法结果大致相同。在大亚湾在线监测的~PQM~图上,有时暗噪声率会超过100~KHz~\ref{fig:darkrate_1},实际的暗噪声率应该在~10 KHz~的数量级上,使用~Rolling Gain~方法给出的暗噪声率为~\ref{fig:darkrate_2},接近真实水平。
\begin{figure}[!htbp]
  \centering
  \begin{subfigure}[b]{\MySubFactor\textwidth}
    \includegraphics[width=\MyFactor\textwidth*5/3]{D:/Thesis/ucasthesis-master/Img/chap3/darkrate1}
    \caption{某段时间~PQM~上错误暗噪声率}
    \label{fig:darkrate_1}
  \end{subfigure}%
  \quad\quad\quad\quad%add desired spacing
  \begin{subfigure}[b]{\MySubFactor\textwidth}
    \includegraphics[width=\MyFactor\textwidth*5/3]{D:/Thesis/ucasthesis-master/Img/chap3/darkrate2}
    \caption{~Rolling Gain~给出的暗噪声率 }
    \label{fig:darkrate_2}
  \end{subfigure}
  \caption{~Rolling Gain~方法分析暗噪声率}
  \label{fig:darkrate}
\end{figure}
\subsection{~PMT~增益与~VME~机箱温度关系研究}
为了搞清楚~PMT~增益与VME机箱温度是否有长期关联性,我们做了该研究。~PMT~的增益数据,通过读数据库中的~CalibPmtFineGain~获得,通过CalibPmtFineGainVld来挑选EH1,AD1的~Rolling~Gain~数据增益,同时需要加入一些质量控制,比如衡量增益拟合优度的量要求小于3等。最终我们挑选了从2011年8月24号至2013年3月16号的2095个整个AD192个~PMT~平均过后的增益数据。~VME~机箱上共有7个温度探头,温度可以通过DCS历史数据获得,我们取得了从2011-07-02到2013-03-15的每天的平均温度。我们将同一天的~VME~机箱温度和~Rolling~ Gain~增益数据作为横纵坐标的一个点填到~Graph~当中,结果如图~\ref{fig:vmeT}~。通过~Excel~分析出机箱温度和~PMT~增益的关联系数为-0.927,负关联明显。

\begin{figure}[!htbp]
  \centering
   \includegraphics[width=\MyFactor\textwidth]{Img/chap3/vmet}
    \caption{~PMT~增益与~VME~机箱温度的关系}
  \label{fig:vmeT}
\end{figure}


\subsection{~PMT~增益漂移一个可能解释}
我们通过~LED~增益刻度和~~刻度均发现大亚湾~PMT~的增益在缓慢上升,大约增加了10\%,见图~\ref{fig:gaindrift}~。
\begin{figure}[!htbp]
  \centering
   \includegraphics[width=\MyFactor\textwidth]{Img/chap3/gaindrift}
    \caption{ 大亚湾~PMT~增益随时间的漂移}
  \label{fig:gaindrift}
\end{figure}




~NIM~上的一篇文章~\citep{aiello2013aging}~,提出的一种理论可以解释这一现象。对于通常的~PMT~双碱光阴极,提出了两种老化理论~\ref{fig:model_1}和~\ref{fig:model_2}。随着~PMT~光阴极的老化,对应着~PMT~阳极灵敏度及输出电流的大小,影响观测到的~PMT~的增益。图~\ref{fig:gain}~展示了随着阳极电荷的积累,单光子波形的电荷幅度的变化。对于一个新的~PMT~来说,其增益是先上升(新的光阴极镀层过厚,随着电子的轰击,镀层变薄,更利于二次电子的发射),后下降的(光阴极结构受损,二次电子发射不足)。为了检验这一理论,我们需要知道大亚湾~PMT~阳极积累电荷的多少。首先通过估算计算。大亚湾~PMT~阳极电流约为100~$\mu A$~,则根据其运行时间,可以推算出阳极总电荷约为7000库伦。按照图~\ref{fig:gain}~的结论,应该处于增益下降阶段。另一种方法是,在我们的探测器中,有一些旧的使用时间比较长的~Marco~PMT,被安装在水池,根据~\ref{fig:marco}~,我们去研究单个~PMT~的增益变化。
\begin{figure}[!htbp]
  \centering
  \begin{subfigure}[b]{\MySubFactor\textwidth}
    \includegraphics[width=\MyFactor\textwidth*3/2]{D:/Thesis/ucasthesis-master/Img/chap3/model1}
    \caption{up-drift mode}
    \label{fig:model_1}
  \end{subfigure}%
  \quad\quad\quad\quad%add desired spacing
  \begin{subfigure}[b]{\MySubFactor\textwidth}
    \includegraphics[width=\MyFactor\textwidth*3/2]{D:/Thesis/ucasthesis-master/Img/chap3/model1}
    \caption{down-drift mode}
    \label{fig:model_2}
  \end{subfigure}
\caption{~PMT~光阴极老化模式}
  \label{fig:model}
\end{figure}



\begin{figure}[!htbp]
  \centering
   \includegraphics[width=\MyFactor\textwidth]{Img/chap3/gain}
    \caption{ ~PMT~增益随阳极总电荷的变化情况}
  \label{fig:gain}
\end{figure}


\begin{figure}[!htbp]
  \centering
   \includegraphics[width=\MyFactor\textwidth]{Img/chap3/marco}
    \caption{~Marco ~PMT~在水池中的分布}
  \label{fig:marco}
\end{figure}



我们检查了60个~marco~PMT~和170个滨松~PMT~增益随时间的变化。结果发现,对于老的~Marco~PMT~来说,其增益保持稳定或者缓慢下降。对于滨松~PMT~来说,其大部分增益上升,只有一小部分保持稳定。图~\ref{fig:gainpmt1}~ 和~\ref{fig:gainpmt2}~分别是同一个channel上滨松和marco ~PMT~的增益变化,其在探测器上的位置接近,因此~hit~occupancy~近似,但明显表现出两种趋势。对于长期运行的中微子实验,这是一种值得注意的现象。


\begin{figure}[!htbp]
  \centering
  \begin{subfigure}[b]{\MySubFactor\textwidth}
    \includegraphics[width=\MyFactor\textwidth*3/2]{D:/Thesis/ucasthesis-master/Img/chap3/b1}
    \caption{}
    \label{fig:gainpmt1_1}
  \end{subfigure}%
  \quad\quad\quad\quad%add desired spacing
  \begin{subfigure}[b]{\MySubFactor\textwidth}
    \includegraphics[width=\MyFactor\textwidth*3/2]{D:/Thesis/ucasthesis-master/Img/chap3/b1}
    \caption{}
    \label{fig:gainpmt1_2}
  \end{subfigure}
   \begin{subfigure}[b]{\MySubFactor\textwidth}
    \includegraphics[width=\MyFactor\textwidth*3/2]{D:/Thesis/ucasthesis-master/Img/chap3/b3}
    \caption{}
    \label{fig:gainpmt1_3}
  \end{subfigure}%
  \quad\quad\quad\quad%add desired spacing
  \begin{subfigure}[b]{\MySubFactor\textwidth}
    \includegraphics[width=\MyFactor\textwidth*3/2]{D:/Thesis/ucasthesis-master/Img/chap3/b4}
    \caption{}
    \label{fig:gainpmt1_4}
  \end{subfigure}
  \caption{滨松~PMT~增益随时间的变化}
  \label{fig:gainpmt1}
\end{figure}

\begin{figure}[!htbp]
  \centering
  \begin{subfigure}[b]{\MySubFactor\textwidth}
    \includegraphics[width=\MyFactor\textwidth*3/2]{D:/Thesis/ucasthesis-master/Img/chap3/m1}
    \caption{}
    \label{fig:gainpmt2_1}
  \end{subfigure}%
  \quad\quad\quad\quad%add desired spacing
  \begin{subfigure}[b]{\MySubFactor\textwidth}
    \includegraphics[width=\MyFactor\textwidth*3/2]{D:/Thesis/ucasthesis-master/Img/chap3/m2}
    \caption{}
    \label{fig:gainpmt2_2}
  \end{subfigure}
   \begin{subfigure}[b]{\MySubFactor\textwidth}
    \includegraphics[width=\MyFactor\textwidth*3/2]{D:/Thesis/ucasthesis-master/Img/chap3/m3}
    \caption{}
    \label{fig:gainpmt2_3}
  \end{subfigure}%
  \quad\quad\quad\quad%add desired spacing
  \begin{subfigure}[b]{\MySubFactor\textwidth}
    \includegraphics[width=\MyFactor\textwidth*3/2]{D:/Thesis/ucasthesis-master/Img/chap3/m4}
    \caption{}
    \label{fig:gainpmt2_4}
  \end{subfigure}
  \caption{Marco~PMT~增益随时间的变化}
  \label{fig:gainpmt2}
\end{figure}


\section{大亚湾中心探测器PMT时间刻度参数检查}
TDC是一种时间幅度转换器件。一旦信号到达该通道并且通过了甄别器设定的阈值,TDC开始计时。接收到触发信号时停止,因此,TDC记录的时间为停止触发信号的时间减去信号开始的时间。大亚湾的PMT的时间刻度主要是时间漂移(timeoffset)刻度和timewalk刻度。对于一个PMT来说:$ T(TDC)=T0+Ttof+T(offset)+T(timewalk) $ 。大亚湾~PMT~目前使用的时间刻度方法为~AD~Timing~。对于一个PMT,使用6个刻度参数来刻度时间,其模型为~\ref{eq:tcor}~。其中x表示电荷。刻度参数通过对图~\ref{fig:tcor}~拟合得到。
\begin{eqnarray}\label{eq:tcor}
f(x) = [0]+[1]exp(-[2]x)+[3]exp(-[4]x)+[5]logx
\end{eqnarray}

\begin{figure}[!htbp]
  \centering
   \includegraphics[width=\MyFactor\textwidth]{Img/chap3/tcor}
    \caption{~PMT~TDC与ADC二维关系}
  \label{fig:tcor}
\end{figure}
我们需要对其时间刻度参数进行检查。利用ACUA ~Z~ 为0位置不同强度的LED刻度run检查,LED驱动电压从-7200~mv~ 到-5400~mv~。定义corrected time $Tcorr=T (TDC)-Ttof-T0-T(offset)-T(timewalk) $ 检查其分布的RMS值得大小。其流程是:
 \begin{enumerate}
\item 使用和LED 发光同步的 B18C16 通道作为参考通道,给出参考时间 
\item 由NuWa下的pmtGeomSvc读PMT几何位置,并根据发光源LED的位置,计算出飞行时间
\item 从数据库TimingCalibData表中读时间刻度参数
\item 由Adc读数计算修正时间
\end{enumerate}
根据图~\ref{fig:tres}~,以及更多不同~LED~光强下,我们看到修正时间的~RMS~一般都小于2ns,因此新的AD Timing刻度参数是非常有效的。

\begin{figure}[!htbp]
  \centering
  \begin{subfigure}[b]{\MySubFactor\textwidth}
    \includegraphics[width=\MyFactor\textwidth*5/3]{D:/Thesis/ucasthesis-master/Img/chap3/tres1}
    \caption{高光强下时间刻度参数检查}
    \label{fig:tres_1}
  \end{subfigure}%
  \quad\quad\quad\quad%add desired spacing
  \begin{subfigure}[b]{\MySubFactor\textwidth}
    \includegraphics[width=\MyFactor\textwidth*5/3]{D:/Thesis/ucasthesis-master/Img/chap3/tres2}
    \caption{ 低光强下时间刻度参数检查}
    \label{fig:tres_2}
  \end{subfigure}
  \caption{~AD Timing~时间刻度参数检查}
  \label{fig:tres}
\end{figure}



\section{中子俘获时间的研究}
\subsection{利用中子俘获时间研究液闪中~Gd~浓度}
掺Gd液体闪烁体探测器长时间运行时,Gd络合物析出的风险。中子可以在H和Gd上俘获。在H上的平均俘获时间和在Gd上的平均俘获时间分别是200微秒和30微秒。因此我们可以通过研究不同时间段的种子俘获时间,来监测不同时间段的Gd的比例。由图~\ref{fig:gds}~可以根据对应的中子俘获时间的变化,得到~Gd~浓度的变化。





\subsection{分析中使用的中子源}
在大亚湾液体闪烁体中,大致存在三种中子的来源。一是:来自IBD反应的中子,一是来自刻度源的中子,另一是来自宇宙线muon引起的spallation 反应带来的中子。
使用的IBD中子来自P12A数据,覆盖时间从2011年9月23好开始2012年7月30号,除了通常的IBD挑选条件外,外加了中子俘获顶点的在探测器中心1m球内的~cut~。以及快慢信号距离小于1m的~cut~。
探测器中存在的刻度中子源有 ~AmC~,~AmBe~,~PuC~中子源。
散列中子由于数目多,并且分布相对均匀,是一种很好的刻度源。散列中子的挑选方法是:
 \begin{enumerate}
 \item 本底时间谱的获得可以通过距离ADmuon后220到400微秒的事例得到
 \item 挑选~ADmuon~后20到200微秒内的事例,需要减去本底的时间谱 
 \item 能量cut为6到12MeV,
 \item 事例顶点在一个半径小于1.4m,高度小于1.4m的圆柱内
  \end{enumerate}
\subsection{中子俘获时间的拟合}
中子俘获时间的拟合通常使用两种方法:单指数+常数,双指数+ 常数。在使用单指数加常数模型时,需要注意的是,时间拟合区间最好从12微秒开始,以避开中子慢化时间的影响。图~\ref{fig:ibdexp_1}~ 和~\ref{fig:ibdexp_2}~分别为单指数拟合和双值拟合情况下,~EH1~AD1~和~AD2~的中子俘获时间。下方面板为AD1和AD2中子俘获时间的差异。IBD中子拟合结果显示,中子俘获时间随时间变化的差异在0.5微秒范围内。

\begin{figure}[!htbp]
  \centering
  \begin{subfigure}[b]{\MySubFactor\textwidth}
    \includegraphics[width=\MyFactor\textwidth*5/3]{D:/Thesis/ucasthesis-master/Img/chap3/ibdexp1}
    \caption{单指数}
    \label{fig:ibdexp_1}
  \end{subfigure}%
  \quad\quad\quad\quad%add desired spacing
  \begin{subfigure}[b]{\MySubFactor\textwidth}
    \includegraphics[width=\MyFactor\textwidth*5/3]{D:/Thesis/ucasthesis-master/Img/chap3/ibdexp2}
    \caption{ 双指数}
    \label{fig:ibdexp_2}
  \end{subfigure}
  \caption{ 单指数和双指数拟合中子俘获时间}
  \label{fig:ibdexp}
\end{figure}
刻度源中子和散裂中子俘获时间拟合结果分别如图~\ref{fig:amc}~ 和~\ref{fig:sn}~ 所示。
得到中子的俘获时间后,就可以拿复活时间来限制Gd的浓度了。根据图~\ref{fig:gds}~的对应关系,从图中可以看出,由于中子俘获时间在很长时间范围内的变化小于0.4微秒,那么我们可以时候响应地Gd浓度变化应该小于0.002\%。并且通过假设检验可以进一步检验在显著性水平0.05 上,中子的俘获时间在误差范围内没有发生变化,也即没有发生~Gd~析出。
\begin{figure}[!htbp]
  \centering
  \begin{subfigure}[b]{\MySubFactor\textwidth}
    \includegraphics[width=\MyFactor\textwidth*5/3]{D:/Thesis/ucasthesis-master/Img/chap3/amc1}
    \caption{单指数}
    \label{fig:amc_1}
  \end{subfigure}%
  \quad\quad\quad\quad%add desired spacing
  \begin{subfigure}[b]{\MySubFactor\textwidth}
    \includegraphics[width=\MyFactor\textwidth*5/3]{D:/Thesis/ucasthesis-master/Img/chap3/amc2}
    \caption{ 双指数}
    \label{fig:amc_2}
  \end{subfigure}
  \caption{ 刻度源中子俘获时间拟合结果}
  \label{fig:amc}
\end{figure}


 \begin{figure}[!htbp]
  \centering
   \includegraphics[width=\MyFactor\textwidth]{Img/chap3/sn}
    \caption{散裂中子俘获时间拟合结果}
  \label{fig:sn}
\end{figure}

\section{新型MCP-PMT测试}
\subsection{后脉冲概念及研究意义}
出现在最初的光电信号一段时间之后的脉冲,如下图~\ref{fig:defination}~ ,幅度可能比主脉冲大,是一种非常讨厌的本底噪声。打拿型~PMT~后脉冲可能的来源:
\begin{enumerate}
\item 从光阴极到第一打拿极的路径上被加速的光电子使PMT内残余气体电离,电离出来的阳离子向光阴极漂,打在光阴极上产生后脉冲。
\item 打拿极上向后散射的电子回到第一打拿极,造成后脉冲。
 \end{enumerate} 
 
 \begin{figure}[!htbp]
  \centering
   \includegraphics[width=\MyFactor\textwidth]{Img/chap3/defination}
    \caption{~PMT~前后脉冲定义}
  \label{fig:defination}
\end{figure}
 
 
  
  图~\ref{fig:afterpulse}~ 为~PMT~DataSheet~ 上的后脉冲数据。对于~DayaBay~ 实验来说,离主脉冲近(几十~ns~)的后脉冲,可能会被~FEE~ 积分进去,从而影响快信号能谱。电子学模拟目前采用的后脉冲模型时间窗口为主脉冲后[500ns, 15000ns], 幅度为单光电子。这与我们的测试结果不同,可以将新的测试结果放入模拟。考虑到未来~JUNO~ 实验采用高速波形采样技术,后脉冲可能会影响能量重建。



 \begin{figure}[!htbp]
  \centering
   \includegraphics[width=\MyFactor\textwidth]{Img/chap3/afterpulse}
    \caption{~PMT~手册上给出的后脉冲数据}
  \label{fig:afterpulse}
\end{figure}





\subsection{测试方法调研}
\subsubsection{~Dayabay~测试方法介绍}
大亚湾实验后脉冲测试使用一个定标器,通过计数主脉冲后100ns-20μs期间的~PMT~的~hit~个数来完成后脉冲测试。测试图为~\ref{fig:dayabay}~。 测试结果为~\ref{fig:dyb_1}~和~\ref{fig:dyb_2}~。左图为测试得到的后脉冲的时间结构,右图为后脉冲率测试结果。该次测试中,后脉冲率定义为~\ref{eq:apr1}~。当主脉冲小于400个pe时,测试的结果为后脉冲率$ APR=0.207+0.017 \times Q $ ,其中Q以pe单位。
\begin{eqnarray}\label{eq:apr1}
after-pulse fraction=\frac{channel 1 count -dark pulse count}{channel 0 count}
\end{eqnarray}

 \begin{figure}[!htbp]
  \centering
   \includegraphics[width=\MyFactor\textwidth]{Img/chap3/dayabay}
    \caption{大亚湾定标器测试~PMT~后脉冲方法}
  \label{fig:dayabay}
\end{figure}



\begin{figure}[!htbp]
  \centering
  \begin{subfigure}[b]{\MySubFactor\textwidth}
    \includegraphics[width=\MyFactor\textwidth*5/3]{D:/Thesis/ucasthesis-master/Img/chap3/dyb2}
    \caption{后脉冲时间结构测试结果}
    \label{fig:dyb_1}
  \end{subfigure}%
  \quad\quad\quad\quad%add desired spacing
  \begin{subfigure}[b]{\MySubFactor\textwidth}
    \includegraphics[width=\MyFactor\textwidth*5/3]{D:/Thesis/ucasthesis-master/Img/chap3/dyb1}
    \caption{后脉冲率测试结果}
    \label{fig:dyb_2}
  \end{subfigure}
  \caption{大亚湾~PMT~测试结果}
  \label{fig:dyb}
\end{figure}


\subsubsection{~Reno~及~KM3NetT~测试方法介绍}
~Reno~测试使用~LED~光源点亮~PMT~,记录300 ns 到15 $\mu$s范围内的PMT Hit的个数。~KM3NetT~实验使用激光点亮~PMT~,使用示波器记录20$\mu$s的信号,其测试示意图及测试结果如图~\ref{fig:km3}~和~\ref{fig:km3r}~所示。
 \begin{figure}[!htbp]
  \centering
   \includegraphics[width=\MyFactor\textwidth]{Img/chap3/km3}
    \caption{~KM3NetT~测试~PMT~后脉冲方法}
  \label{fig:km3}
\end{figure}


\begin{figure}[!htbp]
  \centering
  \begin{subfigure}[b]{\MySubFactor\textwidth}
    \includegraphics[width=\MyFactor\textwidth*5/3]{D:/Thesis/ucasthesis-master/Img/chap3/km3r1}
    \caption{快后脉冲测试结果}
    \label{fig:km3r_1}
  \end{subfigure}%
  \quad\quad\quad\quad%add desired spacing
  \begin{subfigure}[b]{\MySubFactor\textwidth}
    \includegraphics[width=\MyFactor\textwidth*5/3]{D:/Thesis/ucasthesis-master/Img/chap3/km3r2}
    \caption{后脉冲测试结果}
    \label{fig:km3r_2}
  \end{subfigure}
  \caption{~KM3NetT~~PMT~后脉冲测试结果}
  \label{fig:km3r}
\end{figure}

\subsection{基于~Desktop~FADC~的快速后脉冲测试方法}
\subsubsection{高速后脉冲测试方法介绍}
我们使用ps激光器经过衰减后调出单光子或多光子状态。使用激光器的优点是同步信号与光信号时间晃动在几十ps量极,可以准确知道光子到达~pmt~的时间。缺点:~pmt~输出波形的时间会受到~pmt~渡越时间的影响,激光器电源会引入非常大的电磁干扰。测试使用1 ~GHz~采样率的便携式~FADC~。调节PMT高压(~SY1527~高压模块),使PMT增益约为1E7。 信号发生器门信号\~50~ns~宽,PMT的信号在门内,这样可以减少暗噪声的过阈触发。调节信号发生器,型号AFG3102,分别给出单光子,  多光子的信号。另外一种可选择方案是使用示波器,型号~Lecroy~ WavePro ~7100~。 示波器和~CANE~FADC~对比如~\ref{tab:dtwa}~。因此我们最终选择使用DT。

\begin{table}[htbp]
\centering  % 表居中
\begin{tabular}{lcc}  % {lccc} 表示各列元素对齐方式,left-l,right-r,center-c
\hline
测试设备 &DT5751 &WavePro 7100 \\ \hline
\\  采样率&1/(GHz/s) &10/(GHz/s) 
\\  内存 &1.835/Mpts &1/Mpts
\\  死时间& 0$\mu$s &小于6$\mu$s
\\   带宽 &500/MHz &1/GHz
\\   灵敏度(LSB)&1/(mV/div)& 2/(mV/div)
\\  重量(便携性)& 680克 & 18Kg
\\ 价格& 5万元& 15万元
\\ \hline
\end{tabular}
\caption{DT5751与~Lecroy~ WavePro ~7100~对比}
\label{tab:dtwa}
\end{table}

为了准确给出主脉冲以及后脉冲的大小,我们需要做单光子刻度,方法是调节~LD~的~intensity~为单光子, 然后通过拟合得到单光子的数值积分电荷。为了方便分析,我们将事例号,整个20us内波形的个数,每个波形的幅度,面积,开始时间,结束时间,峰值时间记录成~ROOT~Tree~的形式。用于电荷刻度的单光子事例挑选方法是:根据信号产生器门信号的位置,选取第一个~hit~在时间窗口[280ns,330ns]里的~hit~。积分电荷谱的拟合使用与大亚湾Rolling Gain刻度同样的PMT 响应模型。R5912 ~SPE的积分电荷为:0.8559;R5912-100型号 为:0.97817。 单光子电荷刻度结果如图~\ref{fig:speQ}~。
\begin{figure}[!htbp]
  \centering
  \begin{subfigure}[b]{\MySubFactor\textwidth}
    \includegraphics[width=\MyFactor\textwidth*5/3]{D:/Thesis/ucasthesis-master/Img/chap3/speQ1}
    \caption{R5912单光子电荷刻度 测试结果}
    \label{fig:speQ_1}
  \end{subfigure}%
  \quad\quad\quad\quad%add desired spacing
  \begin{subfigure}[b]{\MySubFactor\textwidth}
    \includegraphics[width=\MyFactor\textwidth*5/3]{D:/Thesis/ucasthesis-master/Img/chap3/speQ2}
    \caption{R5912-100 单光子电荷刻度 测试结果}
    \label{fig:speQ_2}
  \end{subfigure}
  \caption{FADC单光子电荷刻度结果}
  \label{fig:speQ}
\end{figure}

\subsubsection{测试结果}
我们的测试结果中后脉冲率的定义为:后脉冲个数/主脉冲Pe数。拟合方程:$N^{afterpulse} = P_0 +P_1\times Q^{mainpulse}$。其中P0 可以认为是PMT的暗计数率  P1即为后脉冲率。如果取中间片段的拟合结果:R5912 的后脉冲率为1.79\% ; R5912-100 的后脉冲率为1.336\%。结果如图~\ref{fig:APRR}~所示。
\begin{figure}[!htbp]
  \centering
  \begin{subfigure}[b]{\MySubFactor\textwidth}
    \includegraphics[width=\MyFactor\textwidth*5/3]{D:/Thesis/ucasthesis-master/Img/chap3/APRR1}
    \caption{R5912后脉冲率 测试结果}
    \label{fig:APRR_1}
  \end{subfigure}%
  \quad\quad\quad\quad%add desired spacing
  \begin{subfigure}[b]{\MySubFactor\textwidth}
    \includegraphics[width=\MyFactor\textwidth*5/3]{D:/Thesis/ucasthesis-master/Img/chap3/APRR2}
    \caption{R5912-100 后脉冲率 测试结果}
    \label{fig:APRR_2}
  \end{subfigure}
  \caption{后脉冲率测试结果}
  \label{fig:APRR}
\end{figure}


后脉冲时间分布也是我们关心的一个内容。图~\ref{fig:APT}~展示了两种~PMT~后脉冲的时间结果。通过上图后脉冲出现时间的分析,可以得出如下结论:R5912,R5912-100内残余气体电离后主要出现两种离子,对R5912,这两种离子造成的后脉冲集中出现在主脉冲后1.7us和6.3us。(大亚湾一期采用multi-hit Tdc测量结果为1.6us 和 7us);对R5912-100,这两种离子造成的后脉冲主要出现在主脉冲后1.5us和7.1us。在我们进行的后脉冲测试中,还存在一种出现的比较快的“后脉冲”,出现在主脉冲后50ns之内。而且非常集中,时间分布的smearing非常小。对R5912,出现比例约为1148/18154=6.3\% ,  对R5912-100,出现的比例约为4900/327601=1.5\%。一种可能的解释:光电子在第一打拿级上被反射(没有出现次级电子发射),回到光阴极后再从新倍增放大。


\begin{figure}[!htbp]
  \centering
  \begin{subfigure}[b]{\MySubFactor\textwidth}
    \includegraphics[width=\MyFactor\textwidth*5/3]{D:/Thesis/ucasthesis-master/Img/chap3/APT1}
    \caption{R5912后脉冲时间分布}
    \label{fig:APT_1}
  \end{subfigure}%
  \quad\quad\quad\quad%add desired spacing
  \begin{subfigure}[b]{\MySubFactor\textwidth}
    \includegraphics[width=\MyFactor\textwidth*5/3]{D:/Thesis/ucasthesis-master/Img/chap3/APT2}
    \caption{R5912-100 后脉冲时间分布}
    \label{fig:APT_2}
  \end{subfigure}
  \caption{后脉冲时间结构}
  \label{fig:APT}
\end{figure}




图~\ref{fig:TQ}~展示了后脉冲的电荷时间而为分布。后脉冲并不仅仅是单光子,从测试结果看还是有很多高于一个1pe 的后脉冲,对R5912-100 尤其明显。
图中用红色框框起来的条状部分,是暗噪声,可见暗噪声在整个时间轴上均匀分布,大小为1个pe。



\begin{figure}[!htbp]
  \centering
  \begin{subfigure}[b]{\MySubFactor\textwidth}
    \includegraphics[width=\MyFactor\textwidth*5/3]{D:/Thesis/ucasthesis-master/Img/chap3/TQ1}
    \caption{R5912后脉冲时间电荷分布}
    \label{fig:TQ_1}
  \end{subfigure}%
  \quad\quad\quad\quad%add desired spacing
  \begin{subfigure}[b]{\MySubFactor\textwidth}
    \includegraphics[width=\MyFactor\textwidth*5/3]{D:/Thesis/ucasthesis-master/Img/chap3/TQ2}
    \caption{R5912-100 后脉冲时间电荷分布}
    \label{fig:TQ_2}
  \end{subfigure}
  \caption{后脉冲时间电荷二维分布}
  \label{fig:TQ}
\end{figure}

通过对比后脉冲测试,FADC测试结果与示波器测试结果基本相同,当信号发生器的脉冲频率为1KHz 时,FADC 触发率约为100Hz ,示波器触发率 仅为1Hz。用DT 开展后续测试更节省时间。另外DT由于可以波形取样,因此通过波形分析,可以得到丰富的物理结果,比如可以得到单光子电荷谱,结果如V965类似,~\ref{fig:QDC_1}~。通过使用皮秒激光器,可以分析~PMT~的渡越时间,比如从图~\ref{fig:QDC_2}~可以看出对于R5912来说,其渡越时间约为2.8ns。



\begin{figure}[!htbp]
  \centering
  \begin{subfigure}[b]{\MySubFactor\textwidth}
    \includegraphics[width=\MyFactor\textwidth*5/3]{D:/Thesis/ucasthesis-master/Img/chap3/QDC1}
    \caption{DT得到的单光子电荷谱}
    \label{fig:QDC_1}
  \end{subfigure}%
  \quad\quad\quad\quad%add desired spacing
  \begin{subfigure}[b]{\MySubFactor\textwidth}
    \includegraphics[width=\MyFactor\textwidth*5/3]{D:/Thesis/ucasthesis-master/Img/chap3/QDC2}
    \caption{DT得到的单光子电荷时间谱}
    \label{fig:QDC_2}
  \end{subfigure}
  \caption{DT采样波形分析的其他结果}
  \label{fig:QDC}
\end{figure}



