
\chapter{光电倍增管的刻度及测试}
\label{chap:chap3}
在中微子实验中,闪烁体或水通常被用于建造大型中微子探测器,需要使用数以万计的光电倍增管~\textt{PMT}~探测闪烁光或切伦科夫光。当光打到置于真空中的金属或者半导体上时,电子就会从其表面发射到真空中,被称为光电效应。%pmt_handbook
光电倍增管就是基于光电效应的一种光电转换器件。由于PMT是一种无损探测器件,并且具有响应速度快,灵敏度高等优点,除了高能物理外,还被广泛运用到医学设备以及工业设备中。

\numberwithin{equation}{section}

\section{大亚湾中心探测器PMT增益刻度}
大亚湾实验中用的PMT增益刻度方法有两种,一种是通过调节驱动~\texttt{LED}~的电压,使~\texttt{LED}~发出单光子,单光子打到~\textt{PMT}~上,从而研究~\textt{PMT}~单光子响应的~\texttt{LED}~ 增益刻度方法。这种方法依赖与~\textt{LED}~的精细调节,而且需要牺牲物理取数时间,专门做这些增益刻度~run~。另一种使用来自非光源的单光电子进行刻度的~\texttt{Rolling Gain}~ 刻度方法。使用的单光电子主要来自~\textt{PMT}~的暗噪声以及随机触发数据。这种方法不需要~\textt{LED}~,而且在物理取数的同时,可以得到刻度数据,提供了~\textt{PMT}~暗噪声率的在线监测方法,对于在线实时处理数据也有好处。做~\texttt{Rolling Gain}~刻度的第一步是刻度数据的挑选。下面是被挑选光子需要满足的一些条件:
\begin{itemize}
\item 距离前一个触发的时间间隔大于20$mu$s
\item 该Hit的peak cycle 在【4,6】之间
\item 每个PMT通道只有一个Hit
\item ADC工作在精细量程区域
\item 电子学基线恢复良好,要求~\left|preAdc-avePed\right|~<20
\item 需要去除40MHz的来自电子学时钟的噪声。方法是:(Tdc-15)\%16>=4
\item TDC位于噪声窗口,及大于1070。另要求噪声窗口内的~hit multiplicity~ 小于3
\end{itemize}
Hit挑选完之后,需要使用~\texttt{PMT}~单光子相应的~Model~去对数据进行~Fit~。 这里使用的模型来自\verb|\citet{BELLAMY1994468}|。在该模型中,光电倍增管被划分为两个相对独立的功能模块。一是模拟光阴极的光子电子转换功能。另一是描述打拿级系统对原初光电子的倍增放大过程。光子转换成光电子的效率由PMT的量子效率决定。考虑到PMT 的收集效率,假设由第一打拿级搜集到的平均光电子数目为~$mu$~,则实际观测到的光电子数目~$n$~与~$mu$~满足泊松分布。$P(n;mu)=\frac{\mu^{n}e^{-\mu}}{n!}$。 电子的倍增放大过程,由于~\texttt{PMT}~自身的电荷分辨,可以用一个高斯分布描述。实际的PMT响应的描述可以表示成如下模式:

这个模型被用到Rolling Gain的Fit当中。一个好的Fit结果需要满足一下条件:
Root Fit的状态为成功。chi2 除以ndf需要小于10. 刻度出来的增益常数在【6.80】 的范围内,并且在边界内。Fit结果的误差需要合理,通常要求小于10。gain 的sigma要求小于gain 值得1/3. gain sigma的绝对值在【1,30】的范围内。
图图




得到刻度常数以后,需要更新刻度参数表。其基本流程如下
首先在本底拷贝一个offline database的副本。往本底副本数据库里插入刻度参数。对本底副本里插入的刻度参数做有效性检验。联系负责人,更新官方离线数据库。
除了提供增益刻度参数外,Rolling Gain还提供了除是用强制触发计算外的另外一种计算方法。暗噪声窗口为1070到1170,共100个TDC,转换为ns为1.5625ns。然后使用暗噪声窗口内的总hit数除以暗噪声窗口长度,即可得到暗噪声率。该方法与强制触发方法结果大致相同。
图图图
\texttt{ucasthesis}宏包可以在目前大多数的~\TeX{}~系统中使用,例如~C\TeX{}、MiK\TeX{}、\TeX{}Live。考虑到大多数用户将是~Windows~ 使用者,推荐安装最新的~C\TeX{}~完整版套装,C\TeX{}~ 完整版套装中包含了本模板中出的各类宏包,用户无需额外的设置即可使用。 \texttt{ucasthesis}~宏包通过~\texttt{ctexbook}~宏包来获得中文支持。
\section{大亚湾中心探测器PMT时间刻度}
TDC是一种时间幅度转换器件。一旦信号到达该通道并且通过了甄别器设定的阈值,TDC开始计时。接收到触发信号时停止,因此,TDC记录的时间为停止触发信号的时间减去信号开始的时间。大亚湾的PMT的时间刻度主要是时间漂移(timeoffset)刻度和timewalk刻度。对于一个PMT来说:$ T(TDC)=T0+Ttof+T(offset)+T(timewalk) $ 由于T(timewalk)较小,所以$T(offset)=T(Tdc)-T0-Ttof$.T0的给出方法有两种:使用reference channel 的读数,或者是所有channel Tdc读数的中位数。\ref{fig:tdc_res}
\begin{verbatim}
\begin{figure}[!htbp]
  \centering
  \begin{subfigure}[b]{\MySubFactor\textwidth}
    \includegraphics[width=\textwidth]{HC_OASPL_A}
    \caption{}
    \label{fig:tdc1}
  \end{subfigure}%
  ~%add desired spacing
  \begin{subfigure}[b]{\MySubFactor\textwidth}
    \includegraphics[width=\textwidth]{HC_OASPL_B}
    \caption{}
    \label{fig:tdc2}
  \end{subfigure}
  \begin{subfigure}[b]{\MySubFactor\textwidth}
    \includegraphics[width=\textwidth]{HC_OASPL_C}
    \caption{}
    \label{fig:tdc3}
  \end{subfigure}%
  ~%add desired spacing
  \begin{subfigure}[b]{\MySubFactor\textwidth}
    \includegraphics[width=\textwidth]{HC_OASPL_D}
    \caption{}
    \label{fig:tdc4}
  \end{subfigure}
  \caption{总声压级。(a)$Tdc offset calibration constants(ns) in DayaBayAD1 by reference channel$,(b)$Tdc offset calibration constants(ns) in DayaBayAD3 by reference channel$,(c)$Tdc offset calibration constants(ns) in DayaBayAD1 by median$,(d)$Tdc offset calibration constants(ns) in DayaBayAD2 by median$}
  \label{fig:tdc_res}
\end{figure}
\end{verbatim}
另外还需要做PMT的时幅漂移修正。$T(timewalk)=T(TDC)-Ttof-T0-T(offset)$。减去T0,Ttof和offset 刻度参数后的得到的calibTime
与Adc的二维Map投影成单值关系,然后做拟合。目前的拟合函数改为:$ y=[0]+[1]*log(x)+[2]*log(x)*log(x)+[3]*log(x)*log(x)*log(x) $ . 一组经验的取值为: p0=-1284.02 p1=558.783 p2=-77.6985 p3=3.59324。

时间刻度参数的检查。利用不同强度的LED刻度run,以及在掺Gd液闪区,和白油区的run来分别检查。定义corrected time $Tcorr=T(TDC)-Ttof-T0-T(offset)-T(timewalk) $ 检查其分布的RMS值得大小。其流程是:
1)获得Adc,Tdc读数(cut条件Tdc>900,<1100)
2) 从数据库中读PMT几何位置,并根据发光源LED的位置,计算出飞行时间(根据DetId和Svcmode)
3) 从数据库中读Tdc Offset参数
4) 由Adc读数计算timewalk(hard code)
5) 获得corrected time 值
经过timeoffset及T0和Ttof修正的结果:

\section{中子俘获时间的研究}
掺Gd液体闪烁体探测器长时间运行时,Gd络合物析出的风险。中子可以在H和Gd上俘获。在H上的平均俘获时间和在Gd上的平均俘获时间分别是200微秒和30微秒。因此我们可以通过研究不同时间段的种子俘获时间,来监测不同时间段的Gd的比例。在大亚湾液体闪烁体中,大致存在三种中子的来源。一是:来自IBD反应的中子,一是来自刻度源的中子,另一是来自宇宙线muon引起的spallation 反应带来的中子。种子俘获时间的拟合通常使用两种方法:单指数+常数,双指数+常数。在使用单指数加常数模型时,需要注意的是,时间拟合区间最好从12微秒开始,以避开中子慢化时间的影响。
图图图



散列中子由于数目多,并且分布相对均匀,是一种很好的刻度源。散列中子的挑选方法是:
挑选ADmuon后20到200微秒内的事例。并且减去本底的时间谱。本底时间谱的获得可以通过距离ADmuon后220到400微秒的事例得到。并且为了得到在纯粹的在Gd上俘获的中,还需要加上能量和有效体积cut。其中能量cut为6到12MeV,有效体积cut为要求事例顶点在一个半径小于1.4m,高度小于1.4m的圆柱内。

得到中子的俘获时间后,就可以拿复活时间来限制Gd的浓度了。其中一个简单的对应关系如下图:从图中可以看出,由于中子俘获时间在很长时间范围内的变化小于0.4微秒,那么我们可以时候响应地Gd浓度变化应该小于0.002\%。并且通过假设检验可以进一步检验在显著性水平0.05 上,中子的俘获时间在误差范围内没有发生变化,也即没有发生Gd析出。



\section{新型MCP-PMT测试}


