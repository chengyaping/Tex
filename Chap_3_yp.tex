
\chapter{光电倍增管的刻度及测试}
\label{chap:chap3}
在中微子实验中,闪烁体或水通常被用于建造大型中微子探测器,需要使用数以万计的光电倍增管~PMT~探测闪烁光或切伦科夫光。当光打到置于真空中的金属或者半导体上时,电子就会从其表面发射到真空中,被称为光电效应。%pmt_handbook
光电倍增管就是基于光电效应的一种光电转换器件。由于~PMT~是一种无损探测器件,并且具有响应速度快,灵敏度高等优点,除了高能物理外,还被广泛运用到医学设备以及工业设备中。
\section{大亚湾中心探测器~PMT~增益刻度}
\subsection{~Rolling Gain~刻度方法介绍}
大亚湾实验中用的~PMT~增益刻度方法有两种,一种是通过调节作为光源的~LED~的驱动电压,使~LED~发出单光子,单光子打到
~PMT~上,从而研究~PMT~单光子响应的~LED~增益刻度方法。这种方法依赖于~LED~的精细调节,而且需要牺牲物理
取数时间,专门做这些增益刻度~run~。另一种使用来自非光源的单光电子进行刻度的~Rolling Gain~刻度方法。使
用的单光电子主要来自~PMT~的暗噪声。这种方法不需要~LED~,而且在物理取数的同时,可以得到
刻度数据,提供了~PMT~暗噪声率的在线监测方法,对于在线实时处理数据也有好处。做~Rolling Gain~刻度的第一
步是刻度数据的挑选。下面是被挑选光子需要满足的一些条件:
\begin{itemize}
\item 距离前一个触发的时间间隔大于20$mu$s
\item 该Hit的peak cycle在【4,6】之间
\item 每个PMT通道只有一个Hit
\item ADC工作在精细量程区域
\item 电子学基线恢复良好,要求$ |preAdc-avePed| <20 $
\item 需要去除40~MHz~的来自电子学时钟的噪声。方法是:(TDC-15)\%16>=4,因~TDC~时钟频率为640~MHz~
\item TDC位于噪声窗口,即大于1070。另要求噪声窗口内的~hit multiplicity~小于3
\end{itemize}
~Hit~挑选完之后,需要使用~PMT~单光子响应模型去对数据进行拟合。这里使用的模型来自
\citep{bellamy1994absolute}。在该模型中,光电倍增管被划分为两个相对独立的功能模块。一是模拟光阴极的光子电
子转换功能。另一是描述打拿级系统对原初光电子的倍增放大过程。光子转换成光电子的效率由~PMT~的量子效率决
 定。考虑到~PMT~的收集效率,假设由第一打拿级搜集到的平均光电子数目为~$\mu$~,则实际观测到的光电子数目
 ~$n$~与~$\mu$~满足泊松分布。$P(n;\mu)=\frac{\mu^{n}e^{-\mu}}{n!}$。 电子的倍增放大过程,由于
 ~PMT~自身的电荷分辨,可以用一个高斯分布描述。实际的PMT响应的描述可以表示成如下模式~\ref{eq:model}~。
 \begin{eqnarray} \label{eq:model}
 S_{real}(x) &=& \int S_{ideal}(x^{\prime})B(x-x^{\prime})dx^{\prime}\\
 S_{ideal}(x) &=& P(n;\mu)\otimes G_n(x) \nonumber \\
 &=& \sum_{n=0}^{\infty} \frac{{\mu}^n e^{-\mu}}{n!}\frac{1}{\sigma_1\sqrt{2n\pi}}exp(-\frac{{(x-nQ_1)}^2}{2n{\sigma_1}^2}) \\
B(x)&=&\frac{1-w}{\sigma_0\sqrt{2\pi}}exp(-\frac{x^2}{2\sigma_0^2})+w\theta (x) \alpha exp(-\alpha x ) \end{eqnarray}
这个模型被用到~Rolling Gain~的拟合当中。一个好的拟合结果需要满足一下条件:
~Root Fit~的状态为成功。~chi2~除以~ndf~需要小于~10~。刻度出来的增益参数在[6,80]的范围内,并且在边界
内。拟合结果的误差需要合理,通常要求小于10。拟合出来的式2.1中的$\sigma$要求小于增益值的1/3,并且在
[1,30]的范围内。一个拟合的示例如图~\ref{fig:rg}~。
\begin{figure}[!htb]
  \centering
   \includegraphics[width=\MyFactor\textwidth]{Img/chap3/rg}
    \caption{~Rolling Gain~PMT~增益刻度结果示例}
  \label{fig:rg}
\end{figure}
得到刻度常数以后,需要更新数据库中的刻度参数表。其基本流程如下,见图~\ref{fig:db}~。
首先在本地拷贝一个离线数据库的副本。往本地副本数据库里插入刻度参数。对本地副本数据库里插入的刻度参数做有效性检验。联系数据库负责人,更新官方离线数据库。
\begin{figure}[!htb]
  \centering
  \begin{subfigure}[b]{\MySubFactor\textwidth}
    \includegraphics[width=\MyFactor\textwidth*5/3]{D:/Thesis/ucasthesis-master/Img/chap3/db1}
    \caption{建立本地副本}
    \label{fig:db_1}
  \end{subfigure}%
  %\quad\quad\quad\quad%add desired spacing
  \begin{subfigure}[b]{\MySubFactor\textwidth}
    \includegraphics[width=\MyFactor\textwidth*5/3]{D:/Thesis/ucasthesis-master/Img/chap3/db2}
    \caption{检验上传}
    \label{fig:db_2}
  \end{subfigure}
  \caption{更新数据库刻度参数操作}
  \label{fig:db}
\end{figure}
除了提供增益刻度参数外,~Rolling Gain~还提供了暗噪声率除强制触发外的另外一种计算方法。暗噪声窗口为~TDC~从1070到1170,共100个TDC,1个~TDC~转换为~ns~为~1.5625~ns。然后使用暗噪声窗口内的总~PMT~击中数除以暗噪声窗口长度,即可得到暗噪声率。该方法与强制触发方法结果大致相同,显示在大亚湾在线监测的~PQM~图上,如图~\ref{fig:darkrate}~所示。
\begin{figure}[!htb]
  \centering
   \includegraphics[width=\MyFactor\textwidth]{Img/chap3/darkrate2}
    \caption{~Rolling Gain~给出的暗噪声率}
  \label{fig:darkrate}
\end{figure}
\subsection{~PMT~增益与~VME~机箱温度关系研究}
为了搞清楚~PMT~增益与~VME~机箱温度是否有长期关联性,我们做了该部分研究。~PMT~ 的增益数据,通过读数据库中的~CalibPmtFineGain~获得,通过~CalibPmtFineGainVld~ 来挑选~EH1~,~AD1~的~Rolling~Gain~数据增益,同时需要加入一些质量控制,比如衡量增益拟合优度的量要求小于3等。最终我们挑选了从2011年8月24号至2013年3月16号的2095个%整个~AD~192个
~PMT~平均增益数据。~VME~机箱上共有7个温度探头,温度可以通过~DCS~历史数据获得,我们取得了从2011-07-02到2013-03-15的每天的平均温度。我们将同一天的~VME~机箱温度和~Rolling~Gain~增益数据作为横纵坐标的一个点填图,结果如~\ref{fig:vmeT}~。 通过~Excel~分析出机箱温度和~PMT~增益的关联系数为-0.927,负关联明显。
\begin{figure}[!htb]
  \centering
   \includegraphics[width=\MyFactor\textwidth]{Img/chap3/vmet}
    \caption{~PMT~增益与~VME~机箱温度的关系}
  \label{fig:vmeT}
\end{figure}
\subsection{~PMT~增益漂移一个可能解释}
我们通过~LED~增益刻度和~Rolling Gain~刻度均发现大亚湾~PMT~的增益在缓慢上升,大约增加了10\%,见图~\ref{fig:gaindrift}~。
\begin{figure}[!htb]
  \centering
   \includegraphics[width=\MyFactor\textwidth]{Img/chap3/gaindrift}
    \caption{ 大亚湾中心探测器和水池~PMT~增益随时间的漂移}
  \label{fig:gaindrift}
\end{figure}
~NIM~上的一篇文章~\citep{aiello2013aging}~,提出一种理论可能解释这一现象。对于通常的~PMT~双碱光阴极,文章提出了两种老化理论~\ref{fig:model_1}和~\ref{fig:model_2}。随着~PMT~光阴极的老化,影响~PMT~阳极灵敏度及输出电流的大小,进而使得观测到的~PMT~的增益变化。图~\ref{fig:gain}~展示了随着阳极电荷的积累,单光子波形的电荷幅度的变化。对于一个新的~PMT~来说,其增益是先上升(新的光阴极镀层过厚,随着电子的轰击,镀层变薄,更利于二次电子的发射),后下降的(光阴极结构受损,二次电子发射不足)。
\begin{figure}[!htb]
  \centering
  \begin{subfigure}[b]{\MySubFactor\textwidth}
    \includegraphics[width=\MyFactor\textwidth*3/2]{D:/Thesis/ucasthesis-master/Img/chap3/model1}
    \caption{up-drift mode}
    \label{fig:model_1}
  \end{subfigure}%
  %\quad\quad\quad\quad%add desired spacing
  \begin{subfigure}[b]{\MySubFactor\textwidth}
    \includegraphics[width=\MyFactor\textwidth*3/2]{D:/Thesis/ucasthesis-master/Img/chap3/model1}
    \caption{down-drift mode}
    \label{fig:model_2}
  \end{subfigure}
\caption{~PMT~光阴极老化模式}
  \label{fig:model}
\end{figure}
\begin{figure}[!htb]
  \centering
   \includegraphics[width=\MyFactor\textwidth*4/5]{Img/chap3/gain}
    \caption{ ~PMT~增益随阳极总电荷的变化情况}
  \label{fig:gain}
\end{figure}
为了检验这一理论,我们需要知道大亚湾~PMT~阳极积累电荷的多少。首先通过估算计算。大亚湾~PMT~阳极电流约为100~$\mu A$~,则根据其运行时间,可以推算出阳极总电荷约为~7000~库伦。按照图~\ref{fig:gain}~的结论,应该处于增益下降阶段。另一种方法是,在我们的探测器中,有一些旧的使用时间比较长的~Macro~PMT,被安装在水池,水池~PMT~信息可参见图~\ref{fig:macro}~, 我们去研究单个~PMT~的增益变化。

\begin{figure}[!htb]
  \centering
   \includegraphics[width=\MyFactor\textwidth]{Img/chap3/marco}
    \caption{~Macro~PMT~在水池中的分布}
  \label{fig:macro}
\end{figure}
我们检查了~60~个~Macro~PMT~和170个滨松~PMT~增益随时间的变化。结果发现,对于老的~Macro~PMT~来说,其增益保持稳定或者缓慢下降。对于滨松~PMT~来说,其大部分增益上升,只有一小部分保持稳定。图~\ref{fig:gainpmt1}~和~\ref{fig:gainpmt2}~分别是同一个~Channel~上滨松~PMT~和~Macro~PMT~的增益变化,其在探测器上的位置接近,因此接收光子情况近似,但明显表现出两种趋势。对于长期运行的中微子实验,这是一种值得注意的现象。
\begin{figure}[!htb]
  \centering
  \begin{subfigure}[b]{\MySubFactor\textwidth}
    \includegraphics[width=\MyFactor\textwidth*5/3]{D:/Thesis/ucasthesis-master/Img/chap3/b1}
    \caption{}
    \label{fig:gainpmt1_1}
  \end{subfigure}%
  %\quad\quad\quad\quad%add desired spacing
  \begin{subfigure}[b]{\MySubFactor\textwidth}
    \includegraphics[width=\MyFactor\textwidth*5/3]{D:/Thesis/ucasthesis-master/Img/chap3/b2}
    \caption{}
    \label{fig:gainpmt1_2}
  \end{subfigure}
   \begin{subfigure}[b]{\MySubFactor\textwidth}
    \includegraphics[width=\MyFactor\textwidth*5/3]{D:/Thesis/ucasthesis-master/Img/chap3/b3}
    \caption{}
    \label{fig:gainpmt1_3}
  \end{subfigure}%
  %\quad\quad\quad\quad%add desired spacing
  \begin{subfigure}[b]{\MySubFactor\textwidth}
    \includegraphics[width=\MyFactor\textwidth*5/3]{D:/Thesis/ucasthesis-master/Img/chap3/b4}
    \caption{}
    \label{fig:gainpmt1_4}
  \end{subfigure}
  \caption{滨松~PMT~增益随时间的变化:增益表现出上升趋势}
  \label{fig:gainpmt1}
\end{figure}

\begin{figure}[!htb]
  \centering
  \begin{subfigure}[b]{\MySubFactor\textwidth}
    \includegraphics[width=\MyFactor\textwidth*5/3]{D:/Thesis/ucasthesis-master/Img/chap3/m1}
    \caption{}
    \label{fig:gainpmt2_1}
  \end{subfigure}%
  %\quad\quad\quad\quad%add desired spacing
  \begin{subfigure}[b]{\MySubFactor\textwidth}
    \includegraphics[width=\MyFactor\textwidth*5/3]{D:/Thesis/ucasthesis-master/Img/chap3/m2}
    \caption{}
    \label{fig:gainpmt2_2}
  \end{subfigure}
   \begin{subfigure}[b]{\MySubFactor\textwidth}
    \includegraphics[width=\MyFactor\textwidth*5/3]{D:/Thesis/ucasthesis-master/Img/chap3/m3}
    \caption{}
    \label{fig:gainpmt2_3}
  \end{subfigure}%
  %\quad\quad\quad\quad%add desired spacing
  \begin{subfigure}[b]{\MySubFactor\textwidth}
    \includegraphics[width=\MyFactor\textwidth*5/3]{D:/Thesis/ucasthesis-master/Img/chap3/m4}
    \caption{}
    \label{fig:gainpmt2_4}
  \end{subfigure}
  \caption{~Macro~PMT~增益随时间的变化,增益下降或稳定}
  \label{fig:gainpmt2}
\end{figure}
\section{大亚湾中心探测器PMT时间刻度参数检查}
TDC是一种时间幅度转换器件。一旦信号到达该通道并且通过了甄别器设定的阈值,~TDC~ 开始计时。接收到触发信号时停止,因此,~TDC~记录的时间为停止触发信号的时间减去信号开始的时间。大亚湾的~PMT~的时间刻度主要是时间漂移(称为~timeoffset~)刻度和~timewalk~刻度。对于一个PMT来说,其记录的~TDC~由事例时间(T0),光子飞行时间(Ttof)以及~PMT~时间刻度参数共同决定:$ T(TDC)=T0+Ttof+T(offset)+T(timewalk) $ 。 大亚湾~PMT~目前使用的时间刻度方法为~AD~Timing~。对于一个~PMT~,使用6个刻度参数来刻度时间,其模型为~\ref{eq:tcor}~。 其中~x~表示该~PMT~纪录的电荷数。刻度参数可通过对图~\ref{fig:tcor}~的拟合得到。
\begin{eqnarray}\label{eq:tcor}
f(x) = [0]+[1]exp(-[2]x)+[3]exp(-[4]x)+[5]logx
\end{eqnarray}
\begin{figure}[!htb]
  \centering
   \includegraphics[width=\MyFactor\textwidth*4/5]{Img/chap3/tcor}
    \caption{~PMT~TDC与~ADC~二维关系}
  \label{fig:tcor}
\end{figure}
为了对~PMT~时间刻度参数进行评估,我们使用了不同光强下的~LED~run~。利用~ACUA~Z~ 为0位置,~LED~驱动电压从-7200~mv~到-5400~mv~。定义修正时间(~corrected~time~)为:$T_{corr}=T(TDC)-Ttof-T0-T(offset)-T(timewalk)$。 检查其分布的~RMS~值的大小。其流程是:
 \begin{enumerate}
\item 使用和~LED~发光同步的~B18C16~通道作为参考通道,给出参考时间T0
\item 由~NuWa~下的~pmtGeomSvc~读PMT几何位置,并根据发光源~LED~的位置,计算出飞行时间
\item 从数据库~TimingCalibData~中读时间刻度参数
\item 由~ADC~读数计算修正时间
\end{enumerate}
修正时间如图~\ref{fig:tres}~所示,我们看到修正时间的~RMS~一般都小于2ns,因此新的~AD Timing~刻度参数是非常有效的,可以认为新刻度参数下,~PMT~的时间分辨率在1~ns~以下。
\begin{figure}[!htb]
  \centering
  \begin{subfigure}[b]{\MySubFactor\textwidth}
    \includegraphics[width=\MyFactor\textwidth*5/3]{D:/Thesis/ucasthesis-master/Img/chap3/tres1}
    \caption{高光强下时间刻度参数检查}
    \label{fig:tres_1}
  \end{subfigure}%
  \quad\quad\quad\quad%add desired spacing
  \begin{subfigure}[b]{\MySubFactor\textwidth}
    \includegraphics[width=\MyFactor\textwidth*5/3]{D:/Thesis/ucasthesis-master/Img/chap3/tres2}
    \caption{ 低光强下时间刻度参数检查}
    \label{fig:tres_2}
  \end{subfigure}
  \caption{~AD Timing~时间刻度参数检查}
  \label{fig:tres}
\end{figure}
%\section{中子俘获时间的研究}
%\subsection{利用中子俘获时间研究液闪中~Gd~浓度}
%掺Gd液体闪烁体探测器长时间运行时,有~Gd~络合物析出的风险。中子可以在~H~和~Gd~ 上俘获。在~H~上的平均俘获时间和在~Gd~上的平均俘获时间分别是200微秒和30微秒\citep{manley1942mean}。因此我们可以通过研究不同时间段的中子俘获时间,来监测不同时间段的~Gd~的比例。由图~\ref{fig:gds}~可以根据对应的中子俘获时间的变化,得到~Gd~浓度的变化。
%\begin{figure}[!htbp]
%  \centering
%   \includegraphics[width=\MyFactor\textwidth]{Img/chap3/gds}
%    \caption{~Gd~浓度与中子俘获时间的关系}
%  \label{fig:gds}
%\end{figure}
%\subsection{分析中使用的中子源}
%在大亚湾液体闪烁体中,大致存在三种中子的来源。一是来自IBD反应的中子,一是来自刻度源的中子,另一是来自宇宙线muon引起的~spallation~反应带来的中子。
%使用的IBD中子来自P12A数据,覆盖时间从2011年9月23号开始到2012年7月30号,除了通常的IBD挑选条件外,外加了中子俘获顶点在探测器中心1m球内的~cut~。以及快慢信号距离小于1m的~cut~。
%探测器中存在的刻度中子源有~AmC~,~AmBe~,~PuC~中子源。
%散列中子由于数目多,并且分布相对均匀,是一种很好的刻度源。散列中子的挑选方法是:
% \begin{enumerate}
% \item 本底时间谱的获得可以通过距离ADmuon后220到400微秒的事例得到
% \item 挑选~ADmuon~后20到200微秒内的事例,需要减去本底的时间谱
% \item 能量cut为6到12MeV,
% \item 事例顶点在一个半径小于1.4m,高度小于1.4m的圆柱内
%  \end{enumerate}
%\subsection{中子俘获时间的拟合}
%中子俘获时间的拟合通常使用两种方法:单指数+常数,双指数+常数。在使用单指数加常数模型时,需要注意的是,时间拟合区间最好从12微秒开始,以避开中子慢化时间的影响。图~\ref{fig:ibdexp_1}~和~\ref{fig:ibdexp_2}~分别为单指数拟合和双值拟合情况下,~EH1~AD1~和~AD2~的中子俘获时间。下方面板为AD1和AD2中子俘获时间的差异。IBD 中子拟合结果显示,中子俘获时间随时间变化的差异在0.5微秒范围内。
%\begin{figure}[!htbp]
%  \centering
%  \begin{subfigure}[b]{\MySubFactor\textwidth}
%    \includegraphics[width=\MyFactor\textwidth*5/3]{D:/Thesis/ucasthesis-master/Img/chap3/ibdexp1}
%    \caption{单指数}
%    \label{fig:ibdexp_1}
%  \end{subfigure}%
%  \quad\quad\quad\quad%add desired spacing
%  \begin{subfigure}[b]{\MySubFactor\textwidth}
%    \includegraphics[width=\MyFactor\textwidth*4/3]{D:/Thesis/ucasthesis-master/Img/chap3/ibdexp2}
%    \caption{ 双指数}
%    \label{fig:ibdexp_2}
%  \end{subfigure}
%  \caption{ 单指数和双指数拟合中子俘获时间}
%  \label{fig:ibdexp}
%\end{figure}
%刻度源中子和散裂中子俘获时间拟合结果分别如图~\ref{fig:amc}~ 和~\ref{fig:sn}~ 所示。
%得到中子的俘获时间后,就可以拿俘获时间来限制Gd的浓度了。根据图~\ref{fig:gds}~ 的对应关系,从图中可以看出,由于中子俘获时间在很长时间范围内的变化小于0.4微秒,那么相应地Gd浓度变化应该小于0.002\%。并且通过假设检验可以进一步检验在显著性水平0.05上,中子的俘获时间在误差范围内没有发生变化,也即没有发生~Gd~析出。
%%\begin{figure}[!htb]
%%  \centering
%%  \begin{subfigure}[b]{\MySubFactor\textwidth}
%%    \includegraphics[width=\MyFactor\textwidth*5/3]{D:/Thesis/ucasthesis-master/Img/chap3/amc1}
%%    \caption{单指数}
%%    \label{fig:amc_1}
%%  \end{subfigure}%
%%  \quad\quad\quad\quad%add desired spacing
%%  \begin{subfigure}[b]{\MySubFactor\textwidth}
%%    \includegraphics[width=\MyFactor\textwidth*5/3]{D:/Thesis/ucasthesis-master/Img/chap3/amc2}
%%    \caption{ 双指数}
%%    \label{fig:amc_2}
%%  \end{subfigure}
%%  \caption{ 刻度源中子俘获时间拟合结果}
%%  \label{fig:amc}
%%\end{figure}
%
% \begin{figure}[!htb]
%  \centering
%   \includegraphics[width=\MyFactor\textwidth]{D:/Thesis/ucasthesis-master/Img/chap3/amc2}
%    \caption{刻度源中子俘获时间拟合结果}
%  \label{fig:amc}
%\end{figure}
%
%
% \begin{figure}[!htb]
%  \centering
%   \includegraphics[width=\MyFactor\textwidth]{Img/chap3/sn}
%    \caption{散裂中子俘获时间拟合结果}
%  \label{fig:sn}
%\end{figure}
\section{基于~Flash-ADC~的~PMT~后脉冲测试系统设计及采样数据分析}
\subsection{后脉冲概念及研究意义}
后脉冲是出现在最初的光电信号一段时间之后的脉冲\citep{morton1967afterpulses},如下图~\ref{fig:defination}~,幅度可能比主脉冲大,是一种非常讨厌的本底噪声。打拿型~PMT~后脉冲可能的来源如下\citep{akchurin2007study}:
\begin{enumerate}
\item 从光阴极到第一打拿极的路径上被加速的光电子使~PMT~内残余气体电离,电离出来的阳离子向光阴极漂,打在光阴极上产生后脉冲。
\item 打拿极上向后散射的电子回到第一打拿极,造成后脉冲。
\end{enumerate}
\begin{figure}[!htb]
  \centering
  \includegraphics[width=\MyFactor\textwidth]{Img/chap3/defination}
  \caption{~PMT~前后脉冲定义}
  \label{fig:defination}
\end{figure}
后脉冲数据是~PMT~DataSheet~上需要给出的~PMT~性能指标之一。在单光子测试,并设定过阈值为1/4单光电子情况下,手册上给出的后脉冲率典型值为2\%,最大值可达10\%。 对于~DayaBay~实验来说,离主脉冲近(几十~ns~)的后脉冲,可能会被~FEE~ 积分电路当做信号积分进去,从而影响快信号能谱。电子学模拟目前采用的后脉冲模型时间窗口为主脉冲后[500ns, 15000ns], 幅度为单光电子。这与我们的测试结果不同,可以将新的测试结果放入模拟。
%\begin{figure}[!htb]
%  \centering
%   \includegraphics[width=\MyFactor\textwidth]{Img/chap3/afterpulse}
%    \caption{~PMT~手册上给出的后脉冲数据}
%  \label{fig:afterpulse}
%\end{figure}
\subsection{测试方法调研}
\subsubsection{~Dayabay~测试方法介绍}
大亚湾实验后脉冲测试使用一个定标器,通过计数主脉冲后100~ns~-20~$\mu$s~期间的~PMT~的~hit~个数来完成后脉冲测试。测试图为~\ref{fig:dayabay}~。 测试结果为,后脉冲的时间分布图在$\sim$1.6~$\mu$s~和$\sim$7~$\mu$s~处会出现两个峰。后脉冲出现几率测试结果为:当主脉冲小于400个~pe~时,后脉冲率为1.7\%$\times$Q~, 其中~Q~以~pe~为单位。
%\begin{eqnarray}\label{eq:apr1}
%after-pulse fraction=\frac{channel 1 count -dark pulse count}{channel 0 count}
%\end{eqnarray}
\begin{figure}[!htb]
  \centering
   \includegraphics[width=\MyFactor\textwidth]{Img/chap3/dayabay}
    \caption{大亚湾定标器测试~PMT~后脉冲方法}
  \label{fig:dayabay}
\end{figure}
%\begin{figure}[!htbp]
%  \centering
%  \begin{subfigure}[b]{\MySubFactor\textwidth}
%    \includegraphics[width=\MyFactor\textwidth*5/3]{D:/Thesis/ucasthesis-master/Img/chap3/dyb2}
%    \caption{后脉冲时间结构测试结果}
%    \label{fig:dyb_1}
%  \end{subfigure}%
%  \quad\quad\quad\quad%add desired spacing
%  \begin{subfigure}[b]{\MySubFactor\textwidth}
%    \includegraphics[width=\MyFactor\textwidth*5/3]{D:/Thesis/ucasthesis-master/Img/chap3/dyb1}
%    \caption{后脉冲率测试结果}
%    \label{fig:dyb_2}
%  \end{subfigure}
%  \caption{大亚湾~PMT~测试结果}
%  \label{fig:dyb}
%\end{figure}
\subsubsection{~KM3NetT~测试方法介绍}
~KM3NetT~实验使用激光点亮~PMT~,使用示波器记录20$\mu$s~内的波形信息,示波器采样率为5GHz,触发工作在~stop~模式。其测试示意图如图~\ref{fig:km3}~ 所示。测试结果观测到的后脉冲的时间分布在$\sim$2~$\mu$s~和$\sim$8~$\mu$s~处会出现两个峰,分别是由于~PMT~内残余的甲烷气体和铯离子造成的。
\begin{figure}[!htb]
  \centering
   \includegraphics[width=\MyFactor\textwidth]{Img/chap3/km3}
    \caption{~KM3NetT~测试~PMT~后脉冲方法}
  \label{fig:km3}
\end{figure}


%\begin{figure}[!htbp]
%  \centering
%  \begin{subfigure}[b]{\MySubFactor\textwidth}
%    \includegraphics[width=\MyFactor\textwidth*5/3]{D:/Thesis/ucasthesis-master/Img/chap3/km3r1}
%    \caption{快后脉冲测试结果}
%    \label{fig:km3r_1}
%  \end{subfigure}%
%  \quad\quad\quad\quad%add desired spacing
%  \begin{subfigure}[b]{\MySubFactor\textwidth}
%    \includegraphics[width=\MyFactor\textwidth*5/3]{D:/Thesis/ucasthesis-master/Img/chap3/km3r2}
%    \caption{后脉冲测试结果}
%    \label{fig:km3r_2}
%  \end{subfigure}
%  \caption{~KM3NetT~~PMT~后脉冲测试结果}
%  \label{fig:km3r}
%\end{figure}

\subsection{基于~Desktop~Flash-ADC~的快速后脉冲测试方法}
\subsubsection{快速后脉冲测试方法介绍}
我们使用皮秒激光器经过衰减后调出单光子或多光子状态。使用激光器的优点是同步信号与光信号时间晃动在几十皮秒量极,可以准确知道光子到达~pmt~的时间。缺点:~pmt~输出波形的时间会受到~pmt~渡越时间的影响,激光器电源会引入非常大的电磁干扰。测试使用1 ~GHz~采样率的便携式~FADC~。调节PMT高压(~SY1527~高压模块),使PMT增益约为1E7。 信号发生器门信号~$\sim$50~ns~宽,PMT 的信号在门内,这样可以减少暗噪声的过阈触发。调节信号发生器,型号AFG3102,分别给出单光子,多光子的信号。另外一种可选择方案是使用示波器,型号~Lecroy~WavePro~7100~。示波器和~CANE~FADC~对比如~\ref{tab:dtwa}~。因此我们最终选择使用~CAEN~公司的~Desktop~Flash-ADC~DT5751。
\begin{table}[htbp]
\centering  % 表居中
\begin{tabular}{lcc}  % {lccc} 表示各列元素对齐方式,left-l,right-r,center-c
\hline
测试设备 &DT5751 &WavePro 7100 \\ \hline
\\  采样率&1/(GHz/s) &10/(GHz/s)
\\  内存 &1.835/Mpts &1/Mpts
\\  死时间& 0$\mu$s &小于6$\mu$s
\\   带宽 &500/MHz &1/GHz
\\   灵敏度(LSB)&1/(mV/div)& 2/(mV/div)
\\  重量(便携性)& 680克 & 18Kg
\\ 价格& 5万元& 15万元
\\ \hline
\end{tabular}
\caption{DT5751与~Lecroy~ WavePro ~7100~对比}
\label{tab:dtwa}
\end{table}
为了准确给出主脉冲以及后脉冲的大小,我们需要做单光子刻度,方法是调节~LD~的~intensity~为单光子, 然后通过拟合得到单光子的数值积分电荷。为了方便分析,我们将事例号,整个20us内波形的个数,每个波形的幅度,面积,开始时间,结束时间,峰值时间记录成~ROOT~Tree~的形式。用于电荷刻度的单光子事例挑选方法是:根据信号产生器门信号的位置,选取第一个~hit~在时间窗口[280ns,330ns]里的~hit~。积分电荷谱的拟合使用与大亚湾Rolling Gain刻度同样的PMT 响应模型。R5912~SPE的积分电荷为:0.8559;R5912-100型号为:0.97817。 单光子电荷刻度结果如图~\ref{fig:speQ}~。
\begin{figure}[!htb]
  \centering
  \begin{subfigure}[b]{\MySubFactor\textwidth}
    \includegraphics[width=\MyFactor\textwidth*5/3]{D:/Thesis/ucasthesis-master/Img/chap3/speQ1}
    \caption{R5912单光子电荷刻度结果}
    \label{fig:speQ_1}
  \end{subfigure}%
  \quad\quad\quad\quad%add desired spacing
  \begin{subfigure}[b]{\MySubFactor\textwidth}
    \includegraphics[width=\MyFactor\textwidth*5/3]{D:/Thesis/ucasthesis-master/Img/chap3/speQ2}
    \caption{R5912-100 单光子电荷刻度结果}
    \label{fig:speQ_2}
  \end{subfigure}
  \caption{FADC单光子电荷刻度结果}
  \label{fig:speQ}
\end{figure}

\subsubsection{测试结果}
我们的测试结果中后脉冲率的定义为:后脉冲个数/主脉冲Pe数。拟合方程:$N^{afterpulse} = P_0 +P_1\times Q^{mainpulse}$。 其中~P0~可以认为是PMT的暗计数率  P1即为后脉冲率。如果取中间片段的拟合结果:~R5912~的后脉冲率为1.79\% ; R5912-100 的后脉冲率为1.336\%。结果如图~\ref{fig:APRR}~所示。
\begin{figure}[!htb]
  \centering
  \begin{subfigure}[b]{\MySubFactor\textwidth}
    \includegraphics[width=\MyFactor\textwidth*5/3]{D:/Thesis/ucasthesis-master/Img/chap3/APRR1}
    \caption{R5912后脉冲率 测试结果}
    \label{fig:APRR_1}
  \end{subfigure}%
  \quad\quad\quad\quad%add desired spacing
  \begin{subfigure}[b]{\MySubFactor\textwidth}
    \includegraphics[width=\MyFactor\textwidth*5/3]{D:/Thesis/ucasthesis-master/Img/chap3/APRR2}
    \caption{R5912-100 后脉冲率 测试结果}
    \label{fig:APRR_2}
  \end{subfigure}
  \caption{后脉冲率测试结果}
  \label{fig:APRR}
\end{figure}
后脉冲时间分布也是我们关心的一个内容。图~\ref{fig:APT}~展示了两种~PMT~后脉冲的时间结果。通过上图后脉冲出现时间的分析,可以得出如下结论:R5912,R5912-100内残余气体电离后主要出现两种离子,对R5912,这两种离子造成的后脉冲集中出现在主脉冲后1.7us和6.3us。(大亚湾一期采用~multi-hit Tdc~测量结果为1.6us 和 7us);对R5912-100,这两种离子造成的后脉冲主要出现在主脉冲后1.5us和7.1us。在我们进行的后脉冲测试中,还存在一种出现的比较快的``后脉冲'',出现在主脉冲后50ns之内。而且非常集中,时间分布的smearing非常小。对R5912,出现比例约为1148/18154=6.3\% ,对R5912-100,出现的比例约为4900/327601=1.5\%。一种可能的解释:光电子在第一打拿级上被反射(没有出现次级电子发射),回到光阴极后再重新倍增放大。
\begin{figure}[!htb]
  \centering
  \begin{subfigure}[b]{\MySubFactor\textwidth}
    \includegraphics[width=\MyFactor\textwidth*5/3]{D:/Thesis/ucasthesis-master/Img/chap3/APT1}
    \caption{R5912后脉冲时间分布}
    \label{fig:APT_1}
  \end{subfigure}%
  \quad\quad\quad\quad%add desired spacing
  \begin{subfigure}[b]{\MySubFactor\textwidth}
    \includegraphics[width=\MyFactor\textwidth*5/3]{D:/Thesis/ucasthesis-master/Img/chap3/APT2}
    \caption{R5912-100 后脉冲时间分布}
    \label{fig:APT_2}
  \end{subfigure}
  \caption{后脉冲时间结构}
  \label{fig:APT}
\end{figure}
图~\ref{fig:TQ}~展示了后脉冲的电荷时间二维分布。后脉冲并不仅仅是单光子,从测试结果看还是有很多高于一个1pe 的后脉冲,对R5912-100 尤其明显。
图中用红色框框起来的条状部分,是暗噪声,可见暗噪声在整个时间轴上均匀分布,大小为1个pe。
\begin{figure}[!htb]
  \centering
  \begin{subfigure}[b]{\MySubFactor\textwidth}
    \includegraphics[width=\MyFactor\textwidth*5/3]{D:/Thesis/ucasthesis-master/Img/chap3/TQ1}
    \caption{R5912后脉冲时间电荷分布}
    \label{fig:TQ_1}
  \end{subfigure}%
  \quad\quad\quad\quad%add desired spacing
  \begin{subfigure}[b]{\MySubFactor\textwidth}
    \includegraphics[width=\MyFactor\textwidth*5/3]{D:/Thesis/ucasthesis-master/Img/chap3/TQ2}
    \caption{R5912-100 后脉冲时间电荷分布}
    \label{fig:TQ_2}
  \end{subfigure}
  \caption{后脉冲时间电荷二维分布}
  \label{fig:TQ}
\end{figure}
通过对比后脉冲测试,FADC测试结果与示波器测试结果基本相同,当信号发生器的脉冲频率为~1~KHz~时,FADC 触发率约为100Hz ,示波器触发率仅为1Hz。用~DT~开展后续测试更节省时间。另外~DT~由于可以波形取样,因此通过波形分析,可以得到丰富的物理结果,比如可以得到单光子电荷谱,结果如V965类似,~\ref{fig:QDC_1}~。通过使用皮秒激光器,可以分析~PMT~的渡越时间,比如从图~\ref{fig:QDC_2}~可以看出对于R5912来说,其渡越时间约为2.8ns。
\begin{figure}[!htb]
  \centering
  \begin{subfigure}[b]{\MySubFactor\textwidth}
    \includegraphics[width=\MyFactor\textwidth*5/3]{D:/Thesis/ucasthesis-master/Img/chap3/QDC2}
    \caption{单光子电荷谱}
    \label{fig:QDC_1}
  \end{subfigure}%
  \quad\quad\quad\quad%add desired spacing
  \begin{subfigure}[b]{\MySubFactor\textwidth}
    \includegraphics[width=\MyFactor\textwidth*5/3]{D:/Thesis/ucasthesis-master/Img/chap3/QDC1}
    \caption{单光子电荷时间谱}
    \label{fig:QDC_2}
  \end{subfigure}
  \caption{~Desktop~FADC~后脉冲测试系统得到的其他结果}
  \label{fig:QDC}
\end{figure}
\section{本章小结}
本章工作内容围绕~PMT~展开。通过对~Rolling~Gain~PMT~增益刻度方法的研究,增加了一种在线监测~PMT~暗噪声率的方法。通过分析~PMT~增益与~VME~机箱温度的关系,发现两者呈现出很强的负关联性,关联系数为-0.927。使用大亚湾外水池中两种不同服役时间的~PMT~,旧的~Macro~PMT~和新的滨松~PMT~,发现~Macro~PMT~增益只呈现下降或稳定趋势,滨松~PMT~增益多数为上升趋势,少数保持稳定。对于长期运行的实验,应该评估该现象对实验的影响。通过定义修正时间,检查大亚湾~AD~Timing~PMT~时间刻度参数,确认该时间刻度方法的时间分辨率在1~ns~以下。

本章另外一部分重要内容为基于~Flash-ADC~的~PMT~后脉冲测试系统的设计及采样数据的分析工作。~Desktop~Flash-ADC~的使用,使得该后脉冲测试系统具有三个优点:一是系统简单,无需使用定标器、甄别器等~NIM~插件;一是通过采样波形分析,可以得到~PMT~ 单光子电荷谱和单光子时间谱等;另外一个是:由于~FADC~触发率可以达到100~Hz~,相比于示波器的1~Hz~左右的触发率,测试时间短。通过对采样波形的数据分析,我们还发现了一种出现在主脉冲后$\sim$50~ns~的``快''后脉冲以及~PMT~后脉冲的电荷分布,目前的电子学模拟不包括``快''后脉冲,而且后脉冲电荷仅为单光子,此测试结果对于电子学模拟具有指导意义。
