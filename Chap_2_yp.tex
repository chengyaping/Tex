
\chapter{中微子物理及中微子探测简介}
\label{chap:chap2}
\section{中微子的发现}
\subsection{~Pauli~提出中微子假说}
~1914~年~James Chadwick~在研究贝塔衰变的能谱时\citep{chadwick1914intensitatsverteilung},发现电子的能谱是连续的,见图~\ref{fig:beta}~。
 当时人们认为贝塔衰变是两体衰变,因此电子的能谱应该是一条单能的线。观测到的连续谱违反了两体衰变过程中的能量和动量守恒。
 1930年12月,~Pauli~提出了存在一种电中性,自旋为1/2的粒子的假说\citep{pauli2letter},并将其称之为中子。该粒子在贝塔衰变中
 和电子一起释放出来,解决了能量和动量不守恒的问题。两年后,中子被发现了,但是由于其质量太大,不能成为贝塔衰变中丢失的中性
 粒子。~1934~年,~E.Fermi~从理论上解释了贝塔衰变\citep{fermi1934versuch},并且给出了贝塔衰变中电子能谱的形状,并且
 将~Pauli~假说的中性粒子改名为中微子。
\begin{figure}[!htbp]
  \centering
   \includegraphics[width=\MyFactor\textwidth]{Img/chap1/beta}
    \caption{镭贝塔衰变中观测到连续的电子能谱}
  \label{fig:beta}
\end{figure}
\subsection{中微子的首次探测}
标准模型中,中微子是自旋为~1/2~的狄拉克(~Dirac~)粒子,静止质量为零,电中
性,无色荷,不参与强相互作用,只参与弱相互作用。
世界上第一个用来探测中微子的探测器是 ``HerrAuge'',见图~\ref{fig:det_1}~。
 坐落在~Hanford~反应堆旁边,是~C. Cowan~和~F. Reines~主导的~Poltergeist~项目的重要组成部分。
 但是这个~1954~年的先驱性的探测器本底远远高于信号。此后,~C. Cowan~和~F. Reines~在~Savannah~河反应堆的旁边
 建造了一个改进的探测器~\ref{fig:det_2}~,见参考文献
\citep{kruse1956detection,reines1960detection},有力的证明了中微子的存在。该探测器中中微子的探测是利用经典的反贝塔衰变:
\begin{eqnarray}\label{eq:ibd}
\bar{\nu_e} + p \rightarrow n + e^+
\end{eqnarray}
\begin{figure}[!htbp]
  \centering
  \begin{subfigure}[b]{\MySubFactor\textwidth}
    \includegraphics[width=\MyFactor\textwidth*6/4]{D:/Thesis/ucasthesis-master/Img/chap1/det1}
    \caption{世界上第一个中微子探测器}
    \label{fig:det_1}
  \end{subfigure}%
  \quad\quad\quad\quad\quad\quad%add desired spacing
  \begin{subfigure}[b]{\MySubFactor\textwidth}
    \includegraphics[width=\MyFactor\textwidth*6/4]{D:/Thesis/ucasthesis-master/Img/chap1/det2}
    \caption{~Savannah~探测器使用的探测原理:中微子的反贝塔衰变}
    \label{fig:det_2}
  \end{subfigure}
  \caption{最早的中微子探测器}
  \label{fig:det}
\end{figure}
~Savannah~河探测器是一个液体闪烁体探测器,中心是掺Cd的水。从反贝塔衰变中出来的正电子与电子发生湮灭,出来一对背对背的伽马,
可以作为快信号。反贝塔衰变中的中子在慢化后被Cd俘获,Cd激发态通过释放伽马退激发,中子俘获产生的伽马作为慢信号。
伽马通过康普顿散射在液闪中激发出闪烁光。闪烁光被探测器四周的光电倍增管探测到。这样通过快慢信号符合给出中微子信号。
多年以后,1995 年~F. Reines~由于首次探测到中微子被授予诺贝尔物理学奖。
$\nu_{\mu}$,第二种轻子味道的中微子,1962年由~L.M. Lederman~, ~M. Schwartz~和~J. Steinberger~在~Brookhaven Alternating Gradient Synchrotron (AGS)~
发现\citep{danby1962observation}。$\nu_{\mu}$ 中微子是由反应$\pi^+ \rightarrow \mu^+ + \nu_{\mu}$ 产生的。1988年出于发现了
第二种中微子
诺贝尔奖被授予这三个人。2000年,$\tau$型中微子在费米实验室的~Direct Observation of Nu Tau (DONUT)~实验上被发现\citep{kodama2001observation}。
在这个实验中,一束800~GeV~的质子束用来轰击钨金属,产生粒子簇射。一部分粒子衰变成$\tau$子。$\tau$子衰变产生$\nu_{\tau}$。
\section{中微子振荡}
中微子振荡的发现是中微子物理对粒子物理和宇宙学产生深刻影响的一个里程碑。
中微子振荡要求中微子拥有不同质量的质量本征态,因此中微子质量不为零。
假设质量最轻的中微子~$\nu_1$~质量为零,则质量差的测量,可以给出中微子质量下限。
本节主要介绍中微子振荡的发现以及中微子振荡理论。
\subsection{中微子振荡的发现}
19世纪60年代后期,物理学家和核化学专家开始致力于对来自太阳的中微子测量。
在太阳内部发生的核聚变反应过程中,~PP~链和~CNO~循环,都会产生大量的电子型中微子\citep{bahcall2005new}。
地球上测量的太阳中微子流强约为~6$\times$10$^{10}$~cm$^{-2}$s$^{-1}$~。图\ref{fig:sun}显示了太阳内部产生的
中微子的能谱。与反应堆相反,太阳内部的~PP~链和~CNO~循环只会产生~$\nu_e$~。
第一个太阳中微子实验,由~RayDavis Jr.~领导的~Homestake~实验,开创了利用放射化学方法探测太阳中微子的先河。
~Homestake~实验利用~600~吨氯乙烯,利用化学反应~Eq.(\ref{eq:clar})~探测来自太阳的$\nu_e$。
\begin{eqnarray}\label{eq:clar}
\nu_e + ^{37}Cl \rightarrow e^- + ^{37}Ar
\end{eqnarray}
\begin{figure}[!htbp]
  \centering
   \includegraphics[width=\MyFactor\textwidth]{Img/chap1/sun}
    \caption{太阳中微子能谱。图中展示了~PP~链(实线)和~CNO~循环中微子的流强(虚线)}
  \label{fig:sun}
\end{figure}
经过几个星期的测量,产生的~$^{37}Ar$~从氯乙烯中分离,~$^{37}Ar$~通过电子俘获成为激发态的~$^{37}Cl$~,
接下来~$^{37}Cl$~通过俄歇电子发射退激发。~Davis~通过正比计数器探测电子来数中微子的个数,
结果发现探测到的中微子比太阳模型预测的要少很多。
其他实验也有类似结果,比如~Gallex/GNO~, ~SAGE~\citep{abdurashitov1994results}和
~Kamiokande~\citep{PhysRevLett.63.16,PhysRevD.44.2241}实验。
这被称为太阳中微子失踪之谜。那么,到底是太阳模型出错了呢,还是中微子在从太阳中心
到达地球的过程中味道发生了改变呢?于此同时,~Super-Kamiokande~\citep{fukuda1998measurement}
实验发现来自大气的~$\nu_{\mu}$~也比理论预期的少。大气中微子主要通过~$\pi$~介子衰变 ,
以及后续的~$\mu$~介子衰变产生。这样~$\nu_{\mu}$~与~$\nu_{\mu_e}$~的数目比应当为2:1。
有趣的是,~Super-Kamiokande~实验还发现来自地球下方的大气中微子偏离预期值最远\citep{PhysRevLett.82.2644}。
为了研究太阳中微子丢失之谜,加拿大~SNO~实验\citep{mcdonald1999sudbury}使用~1000~吨重水去探测各种味道的中微子。
重水中可以发生如下反应:
\begin{eqnarray}\label{eq:sno}
&\nu_x +e  \rightarrow \nu_x + e^- \\
& \text{(所有味道中微子都可以发生NC过程,只有$\nu_e$可以发生CC过程)} \nonumber \\
&\nu_e +d \rightarrow p + p + e^- \\
& \text{(只有$\nu_e$可以发生CC过程)} \nonumber \\
&\nu_x +d  \rightarrow p + n + \nu_x \\
& \text{(所有味道中微子都可以发生NC过程)} \nonumber
\end{eqnarray}
~SNO~实验可以测量各种味道的中微子的总流强和电子中微子的流强。
为了验证和提高中微子流强的测量精度,SNO实验前后使用了~3~种方案,
第一期使用重水,第二期使用盐水\citep{ahmed2004measurement},
第三期使用$^{3}He$来俘获中性流产生的中子\citep{amsbaugh2007array}。
~SNO~实验证实了来自太阳的中微子流强是守恒的\citep{ahmad2001measurement},
只是从太阳到地球的过程中,中微子的味道发生了变化,
图~\ref{fig:sno}~显示~SNO~实验三种中微子流强的测量结果。
\begin{figure}[!htbp]
  \centering
   \includegraphics[width=\MyFactor\textwidth]{Img/chap1/sno}
    \caption{~SNO~实验给出的$^{8}B$中微子各种味道的流强}
  \label{fig:sno}
\end{figure}
\subsection{中微子振荡的理论描述}
中微子振荡的现象起源于中微子的质量本征态和味道本征态不一致,
也就是参与弱相互作用的中微子态与传播过程中的中微子态不一样。
这个混合机制可以和夸克混合相类比。不过由于色禁闭,
我们不能探测到自由夸克。
中微子味道本征态~$\nu_e$~,~$\nu_{\mu}$~,~$\nu_{\tau}$~,
可以看做是质量本征态~$\nu_1$~,~$\nu_2$~,~$\nu_3$~的如下叠加。
\begin{eqnarray}
\left(\begin{array}{c}
\nu_e \\ \nu_{\mu} \\  \nu_{\mu}
 \end{array} \right)
 =\left(\begin{array}{ccc}
      1 & 0 & 0 \\ 0 & C_{23} & S_{23} \\ 0 & -S_{23} & C_{23}
      \end{array} \right)
 \left(\begin{array}{ccc}
    e^{i\phi_1}  &  &    \\
    &  e^{i\phi_2}  &    \\
    &               & 1    \end{array} \right)
\label{eq:linearMix1}
\end{eqnarray}
其中~U~被称作是~Pontecorvo-Maki-Nakagawa-Sakata~(PMNS)矩阵。
它包含了3个混合角~$\theta_{ij}$~和1个复的狄拉克相角~$\delta_D$~。
另外还有两个复的~Majorana~相角。~PMNS~矩阵通常被写成如下形式:
\begin{eqnarray} U
 =\left(\begin{array}{ccc}
      1 & 0 & 0 \\ 0 & C_{23} & S_{23} \\ 0 & -S_{23} & C_{23}
      \end{array} \right)
 \left(\begin{array}{ccc}
      C_{13} & 0 & S_{13}e^{-i\delta_D} \\ 0 & 1 & 0 \\
      -S_{13}e^{-i\delta_D} & 0 & C_{13}
      \end{array} \right)
 \left(\begin{array}{ccc}
      C_{12} & S_{12} & 0 \\ -S_{12} & C_{12} & 0 \\ 0 & 0 & 1
      \end{array} \right)
 \left(\begin{array}{ccc}
    e^{i\delta_{M1}}  & 0 &  0  \\
   0 & e^{i\delta_{M2}}  & 0   \\
    0&     0          & 1    \end{array} \right)
\label{eq:MMatrix}
\end{eqnarray}
其中$S_{ij} = sin (\theta_{ij}) $和$C_{ij} = cos (\theta_{ij}) $。
中微子质量本征态的演化遵循薛定谔方程。
假设时刻0产生了一个电子型中微子,那么0时刻的状态方程可以写作:
\begin{eqnarray}\label{eq:eq1}
|\nu\big( t>0 \big ) \rangle = | \nu_e\rangle =U^*_{e1}|\nu_1\rangle+U^*_{e2}|\nu_2\rangle+U^*_{e3}|\nu_3\rangle .
\end{eqnarray}
中微子在空间中的传播由动量为$P_i$和能量为$E_i$的质量本征态决定。
那么一段时间~t~后,中微子的状态可以写成:
\begin{eqnarray}\label{eq:eq2}
|\nu\big (t>0\big )\rangle = | \nu_e\rangle =U^*_{e1}e^{-iE_1t}|\nu_1\rangle+U^*_{e2}e^{-iE_2t}|\nu_2\rangle+U^*_{e3}e^{-iE_3t}|\nu_3\rangle\neq |\nu_e\rangle .
\end{eqnarray}
这种质量本征态的叠加并不一定是味道本征态。
因此,在0时刻后的某一时刻,测量到另一种味道的中微子的几率非0。
我们只能探测到中微子的味道本征态。因为质量本征态可以看做是味道本征态的叠加,
因此我们可以把公式~\ref{eq:eq2}~写成如下形式
\begin{eqnarray}\label{eq:eq3}
 | \nu_{\alpha}(t)\rangle &=&\sum_k U^*_{\alpha k}e^{-iE_k t}|\nu_k\rangle \\
 &=& \sum_{\beta=e,\mu,\tau} \left ( \sum_k U^*_{\alpha k}e^{-iE_k t}U^*_{\beta k} \right )| \nu_{\beta}\rangle\\
| \nu_{k}\rangle &=& \sum_{\beta =e ,\mu,\tau}U_{\beta k}|\nu_{\beta} \rangle
\end{eqnarray}
探测到~$\nu_{\beta}$~态的几率可以由~$|\nu_{\alpha} (t) \rangle $~态在味道本征态~$\nu_{\beta}$~上的投影得到。表示如下
\begin{eqnarray}\label{eq:eq4}
P(\nu_{\alpha\rightarrow \beta }(t))={\langle \nu_{\beta}| \nu_{\alpha}(t)\rangle}^2={A_{\nu_{\alpha} \rightarrow \nu_{\beta}}(t)}^2={\sum_k U^*_{\alpha k}e^{-iE_k t}U_{\beta k}}^2 \\
=\sum_{k,j}U^*_{\alpha k}U_{\beta k}U_{\alpha j}U^*_{\beta j}e^{-i(E_k-E_j)t}
\end{eqnarray}
使用极端相对论近似,$p_k = p = E$,公式~\ref{eq:eq4}~可以写成如下形式:
\begin{eqnarray}\label{eq:eq5}
P(\nu_{\alpha\rightarrow \beta }(L/E))=
\sum_{k,j}U^*_{\alpha k}U_{\beta k}U_{\alpha j}U^*_{\beta j}e^{-i\frac{\Delta{m_{ij}^2}L}{2E}}
\end{eqnarray}
其中$\Delta_{m_{ik}^2}=m_i^2-m_j^2$对应质量平方差。~L~表示探测器和中微子源的距离,
~E~表示中微子的能量。从公式~\ref{eq:eq5}~可以看出,
振荡几率是由~PMNS~矩阵混合参数和重量平方差决定的。
由于混合角~$\theta_{13}$~比较小,
通常的3代中微子混合可以简化成2种味道的中微子的混合,
如公式~\ref{eq:eq6}~所示。
\begin{eqnarray}\label{eq:eq6}
P_{\nu_e  \rightarrow \nu_{\mu} } (L/E)=sin^2(2\theta)sin^2\left(\frac{\Delta m^2L}{2E}\right)
\end{eqnarray}
从公式~\ref{eq:eq6}~中我们可以看到中微子振荡的振幅由$sin^2(2\theta)$决定,
中微子振荡的频率由$\Delta m^2 $决定。这是实验中经常用到的一个中微子产生几率的公式。
\section{中微子探测}
\subsection{中微子与物质相互作用}
中微子没有电荷,不参与电磁相互作用;
中微子没有色荷,不参与强相互作用;
中微子只参与弱相互作用。
实验中观测中微子,通常有以下几种相互作用:
中微子与粒子交换~W$^{\pm}$~的
相互作用为带电流(CC)相互作用,如图~\ref{fig:cc}~,
交换~Z$^0$~的相互作用为中性流(NC)相互作用图~\ref{fig:nc}~,
各种味道的中微子与核子的中性流相互作用过程截面相同。
弹性散射可以是带电流过程,~$\nu_e+l \rightarrow \nu_e+l $~如图~\ref{fig:es}~左;
也可以是中性流过程,~$\nu+l \rightarrow \nu+l$如图~\ref{fig:es}~右。
仅有~$\nu_e$~可以通过~NC~和~CC~过程与电子发生弹性散射,
因此在中微子电子弹性散射截面中,~$\nu_e$~截面最大。
\begin{figure}[!htb]
  \centering
  \begin{subfigure}[b]{\MySubFactor\textwidth}
    \includegraphics[width=\MyFactor\textwidth*6/4]{D:/Thesis/ucasthesis-master/Img/chap1/cc}
    \caption{带电流过程}
    \label{fig:cc}
  \end{subfigure}%
  %\quad\quad\quad\quad\quad\quad%add desired spacing
  \begin{subfigure}[b]{\MySubFactor\textwidth}
    \includegraphics[width=\MyFactor\textwidth*5/4]{D:/Thesis/ucasthesis-master/Img/chap1/nc}
    \caption{中性流过程}
    \label{fig:nc}
  \end{subfigure}
  \caption{中微子与物质相互作用的带电流和中性流过程}
  \label{fig:ccnc}
\end{figure}


%\begin{figure}[!htbp]
%  \centering
%   \includegraphics[width=\MyFactor\textwidth*2/3]{Img/chap1/cc}
%    \caption{ 中微子与物质相互作用的带电流过程}
%  \label{fig:cc}
%\end{figure}
%\begin{figure}[!htbp]
%  \centering
%   \includegraphics[width=\MyFactor\textwidth*2/3]{Img/chap1/nc}
%    \caption{ 中微子与物质相互作用的过程}
%  \label{fig:nc}
%\end{figure}
\begin{figure}[!htb]
  \centering
   \includegraphics[width=\MyFactor\textwidth]{Img/chap1/es}
    \caption{ 中微子与物质相互作用的弹性散射过程}
  \label{fig:es}
\end{figure}

\subsection{中微子探测方法}
中微子的探测通常使用三种技术:切伦科夫辐射探测,放射化学方法探测,液体闪烁体探测。
\subsubsection{(重)水切伦科夫探测器}
(重)水切伦科夫探测器,被广泛应用于多个中微子领域的研究。
比如太阳中微子(~Kamiokande,Superkamiokande,SNO~);
大气中微子(~Kamiokande,Superkamiokande~);
加速器中微子(~Kamiokande,Superkamiokande~);
极高能中微子(~AMANDA\citep{halzen1998amanda},Baikal\citep{belolaptikov1997baikal}~)。

以~SNO~实验为例介绍中微子切伦科夫探测器。
~SNO~实验是一个重水切伦科夫探测器。
位于加拿大,安大略湖省,萨德伯里市
克赖顿矿地下2000~m~的洞穴里\citep{ewan1992sudbury}。
~SNO~使用1000吨装在直径12~m~的亚力克球中的重水作为探测器靶物质。
中微子与重水反应产生切伦科夫辐射,被周围9600个光电倍增管收集到。

~SNO~实验的物理目标是独立测量来自太阳的电子型中微子的流强以及谬型和套型中微子的总流强。
中微子在重水中与物质的相互作用有三种:带电流,中性流,
电子弹性散射 \citep{ahmad2001measurement,jillings1999electron,ying1992charged}。
带电流过程可表示为$\nu_e+D \rightarrow p + p + e^- $。
当中微子靠近重水中的一个氘核时,通过交换~W~波色子,将氘核里的一个中子转化为质子,
中微子转换成电子。
在这个过程中,中微子的能量几乎全部转换成电子的能量。
电子在重水中发生切伦科夫辐射,被光电倍增管探测到。
根据标准模型的计算,在~SNO~探测器中,一天约有30个带电流事例。

重水中的中性流过程可以表示为$\nu_x + D \rightarrow p + n + \nu_x $。
在这个过程当中,发生了~Z~波色子的交换,
一个氘核被打碎成一个中子和一个质子,
中子在重水中通过散射慢化。
慢化后的中子被俘获,产生伽马射线,伽马光子与电子发生康普顿散射,
电子的切伦科夫辐射可以被光电倍增管探测到。
各种类型的中微子发生中性流过程的截面完全相同。探测效率取决于中子的俘获效率以及
后续的伽马光子的探测效率。中子可以直接被氘核俘获,但是效率比较低。
在~SNO~实验运行的第二阶段,往重水中加入了NaCl,参见文献\citep{ahmed2004measurement},
中子在$^{35}$Cl~上的俘获(如图~\ref{fig:snonc}~所示)效率要高很多。
根据标准模型的计算,一天约有30个中性流事例。
\begin{figure}[!htb]
  \centering
   \includegraphics[width=\MyFactor\textwidth*2/3]{Img/chap1/snonc}
    \caption{加入NaCl后的重水中的中性流过程}
  \label{fig:snonc}
\end{figure}

重水中的电子弹性散射过程可以表示为
$\nu_x + e^- \rightarrow \nu_x +e^- $。
该反应道可以探测所有类型的中微子,
但对于电子型中微子截面要高出6倍。
根据标准模型的预测,~SNO~探测器中一天约有3个电子弹性散射事例。

~SNO~实验由于使用了重水,可以根据$F(\nu_{\mu} + \nu_{\tau})=F(\nu_x)-F(\nu_e)$
来确定谬型和套型中微子的总流强。
实验结果如图~\ref{fig:sno}~所示\citep{ahmad2002direct,aharmim2013combined}。

\subsubsection{基于放射化学方法的中微子探测器}
基于放射化学方法的实验有~Homestake~实验,~Gallex/GNO~实验和~SAGE~实验。
最早的~Homestake~实验基于式~\ref{eq:clar}~所示的反应原理。
中微子的能量阈值为814~KeV~\citep{cleveland1995update}。
该方法由~Pontecorvo~在1946年提出\citep{pontecorvo1946chalk},
1949年,~Alvarez~也独立提出了该探测方法 \citep{alvarez1949university}。
使用~Ga~的实验基于反应~\ref{eq:ga}~, 中微子能量阈值为~233~KeV~。
最早是由~Kuzmin~在~1965~年提出的\citep{kuzmin1965solar}。
\begin{eqnarray}\label{eq:ga}
\nu_e + ^{71}Ga \rightarrow ^{71}Ge + e^-
\end{eqnarray}
放射化学方法被广泛用于太阳中微子的研究。
我们将~GNO~实验\citep{collaboration1996proposal}作为例子,
详细介绍中微子探测的放射化学探测方法。
~GNO~实验的物理目标是测量低能太阳中微子流强。
该实验位于意大利~GranSasso~国家实验室。
根据标准太阳模型,~Galliium~探测器探测到的太阳中微子,
53\%来自pp中微子,27\%来自$^7$Be中微子,
12\%来自$^8$B中微子,8\%来自~CNO~中微子\citep{bahcall2001solar},
见图~\ref{fig:gno1}~。
\begin{figure}[!htb]
  \centering
   \includegraphics[width=\MyFactor\textwidth]{Img/chap1/gno1}
    \caption{ 标准太阳模型下中微子反应率}
  \label{fig:gno1}
\end{figure}
探测器的靶物质是101吨~GaCl$_3$~的水和~HCl~溶液,如图~\ref{fig:gallex}~。
其中约有30.3吨Ga元素,~$^{71}$Ga~的数目约为10$^{29}$个。
\begin{figure}[!htb]
  \centering
   \includegraphics[width=\MyFactor\textwidth*3/5]{Img/chap1/gallex}
    \caption{~GNO/Gallex~实验:101吨~GaCl$_3$~的~HCl~溶液}
  \label{fig:gallex}
\end{figure}
反应~\ref{eq:ga}~产生的~$^{71}$Ge~可以通过电子俘获重新生成~$^{71}$Ga~。
~$^{71}$Ge~的平均寿命约为16天。反应~\ref{eq:ga}~和~$^{71}$Ge~的电子俘获之间可以达到动态平衡。
处于反应阈值(~233~KeV~)以上的太阳中微子的流强可以根据理论计算出来的反应截面
通过~$^{71}$Ge~的数目推导出来。~$^{71}$Ge~可以通过化学方法从溶液中分离,
然后通过计数衰变次数数出其个数\citep{bahcall1997gallium}。
~GNO~实验测量太阳中微子流强的步骤可以总结如下:
\begin{itemize}
\item 将氯化镓溶液暴露在太阳中微子下4个星期,溶液中通过反应~\ref{eq:ga}~约有10个~$^{71}$Ge~产生。
\item ~$^{71}$Ge~在溶液中以~GeCl$_4$~的形式存在,可以通过往溶液中通入约3000立方米氮气萃取到水中\citep{henrich1997angew}。
\item 提取的~$^{71}$Ge~被转变成~GeH$_4$~。被送入以氙气为工作气体的正比计数器中计数。萃取和转换的效率约为95-98\%\citep{wink1993miniaturized}。
\item 在正比计数器中,~$^{71}$Ge~通过电子俘获衰变。为了保证~$^{71}$Ge~完全衰变以及搞清楚计数器的本底,整个计数过程持续6个月。~$^{71}$Ge~衰变产生的电脉冲被一个快速的采样系统以0.2~ns~每通道的速度采样,采样时间窗为400ns。
\item 通过采样波形分析,以及一个基于极大似然拟合的方法获取计数器中的~$^{71}$Ge~数目。
\end{itemize}
~GALLEX~实验取数时间为1991年至1997年,
一共有65个取数周期。
太阳中微子在~$^{71}$Ga~的反应率为:\\
\centerline{~GALLEX~:77.5 $\pm$ 6.2 (stat.) +4.3 -4.7 (sys.) SNU \citep{hampel1999gallex}}\\
其中~1~SNU~表示每秒在单个~$^{71}$Ga~上10$^{-36}$个反应。
此后,该实验进行了电子学及~DAQ~升级,
并对化工厂进行了维护,从1998年4月份重新开始取数,
实验改名为~GNO~。从1998年5月至2002年1月,~GNO~一共进行了43个取数周期,
1240天太阳中微子的曝露时间,总共收集到大约200个~$^{71}$Ge~衰变事例。
太阳中微子反应速率为:\\
 \centerline{~GNO~:65.2 $\pm$ 6.4 (stat.) +/- 3.0  (sys.) SNU \citep{kirsten2002talk}}\\
~GALLEX~实验和~GNO~实验的联合分析结果为:(使用65+35=100个取数周期,1594+1240天曝露时间)\\
   \centerline{~GNO~+~GALLEX~:70.8 $\pm$ 4.5 (stat.) +/- 3.8 (sys.) SNU  \citep{kirsten2002talk}}
\subsubsection{液体闪烁体中微子探测器}
液体闪烁体探测器探测$\bar{\nu_e}$通常基于
反贝塔衰变~$\bar{\nu_e} + p \rightarrow e^+ + n $~,
能量阈值是~1.8~MeV~,这是一个带电流过程,
截面大概在~10$^{-42}cm^2$~量级。
探测~$\nu_e$~通常使用
$\nu_x +e ^- \rightarrow \nu_x +e ^-$这一反应道。
这是一个弹性散射过程,能量阈值是~200~KeV~,
对所有味道的中微子均敏感。但电子类型的中微子截面要高出其他类型中微子6个数量级。
液体闪烁体探测器实验的研究对象涵盖太阳中微子(可以检验标准太阳模型),
反应堆中微子,
地球中微子(地球中84\%的热量来自~U~Th~天然衰变链,伴随产生中微子)。
液体闪烁体探测器具有对低能、短射程粒子有比较高的探测效率的优点\citep{gibson1971liquid}。

粒子在闪烁体中的发光过程如下\citep{lakowicz2013principles}:
粒子的能量绝大部分先传递给溶剂,
引起溶剂分子的电离和激发。
大部分受激发分子不参与闪烁过程,
以热能的形式失去能量;
其中部分激发的溶剂分子处于高能态,
当其迅速地退激时,
便将能量传递给周围的第一闪烁剂 (~primary~scintillator~)(液闪中的发光物质被称为闪烁剂(~scintillator~)),
使之受激发。受激发的高能态闪烁剂分子退激复原时,能量发生转移,在瞬间发射出光子。
闪烁液中溶剂分子占99\%以上,
闪烁剂分子的浓度一般在1\%以下。
各种第一闪烁剂分子固有的发光光谱各不相同。
对于大亚湾液闪来说,闪烁剂为PPO,
其发射光谱峰值为376~nm~,
能很好的溶解于LAB。
有时为了与光电倍增管的光电阴极响应光谱相匹配,
需加入第二闪烁剂(~secondary~scintillator~),
使之与所用的光电倍增管的光阴极波长相匹配。
对于大亚湾实验来说,使用的第二闪烁剂(或者称为波长移位剂)为~Bis-MSB~,
其发射光谱峰值为416~nm~。
闪烁过程中能量转换的任何一步,
其效率都不是100\%的。
处于激发态的溶剂分子(S*)或其二聚体(SS*)
(S*:激发态的溶剂分子)、闪烁剂分子(F*)或其二聚体(FF*)
(F*:激发态的闪烁剂分子)
恢复到基态时,总有一部分能量以非光子形式
(以热为主)损失掉,
已经形成的光子也可被闪烁液内某些成分吸收,
不能到达光电倍增管。
上述各种方式的能量损失,统称为淬灭(~quenching~)。
比如闪烁剂分子浓度过高时容易形成二聚体,称浓度淬灭,
另外溶质本身浓度太高也会引起自吸收。
详细介绍可以在文献\citep{birks2013theory}中找到。

由于本论文研究基于大亚湾实验和江门中微子实验,因此选取他们作为闪烁体探测器的代表,一一介绍。

大亚湾实验位于广东省深圳市。主要目标是利用核反应堆产生的电子反中微子
来测定一个具有重大物理意义的参数-中微子混合角$\theta_{13}$
\citep{pontecorvo1968neutrino,maki1962remarks}。
大亚湾核电站与岭澳核电站,
共有六个反应堆,见图~\ref{fig:dyb}~,总热功率为17.4~GW~。
紧靠反应堆有高山,在距反应堆约2公里处,
山体覆盖高度约为400米,近点探测器,山体覆盖100米以上。
山体由整体的花岗石构成,很适于隧道开凿和建立较大的地下实验室。
从隧道入口处到大亚湾近点实验站则采用有坡度的隧道,可以将探测器置于更深的地下,减小宇宙线本底的影响。
\begin{figure}[!htb]
  \centering
   \includegraphics[width=\MyFactor\textwidth]{Img/chap1/dyb}
    \caption{大亚湾反应堆中微子实验的布局示意图}
  \label{fig:dyb}
\end{figure}
3个地下实验大厅,共有8个全同的反中微子探测器。
反中微子探测器是三层同心圆柱结构\citep{an2012side}。
最里层为掺有0.1\%~Gd~的液闪\citep{yeh2007gadolinium}。
中心为普通液闪,最外层为白油。探测原理为反贝塔衰变:
$\nu_e + p \rightarrow e^+ + n $。正电子被称为快信号。
中子慢化后被~Gd~\citep{an2015new}或者~H~元素\citep{an2014independent}俘获,
产生的伽马作为慢信号,如图~\ref{fig:nhngd}~所示。
\begin{figure}[!htb]
  \centering
   \includegraphics[width=\MyFactor\textwidth]{Img/chap1/nhngd}
    \caption{大亚湾中微子IBD反应快慢信号示意图}
  \label{fig:nhngd}
\end{figure}
通过对大亚湾404天数据,远点超过15万~IBD~事例的分析,
与无振荡假设预期的结果对比,给出sin$^2$(2$\theta_{13}$)的最佳拟合结果为0.084$\pm$0.005。
$|\Delta^2_{ee}|$的最佳结果为 (2.42$\pm$0.11)$\times$10$^{-3}$ eV$^2$。
大亚湾实验的远近点设置,
可以开展惰性中微子的研究\citep{an2014search}。在${|\Delta m_{14}|}^2<0.1$范围内,可以对$\theta_{14}$提供非常好的限制。
\begin{figure}[!htb]
  \centering
   \includegraphics[width=\MyFactor\textwidth]{Img/chap1/resultdyb}
    \caption{大亚湾 6.9$\times$10$^5$ GW$_{th}$-ton-days Exposure sin$^2$(2$\theta_{13}$) 及$|\Delta^2_{ee}|$拟合结果}
  \label{fig:resultdyb}
\end{figure}
江门中微子实验的首要物理目标是通过精确测量反应堆中微子的能谱来确定中微子质
量顺序\citep{magg1980neutrino} (见图~\ref{fig:mhh0}~) 以及精确测量中微子的振荡参数。
此外,还进行其它多项前沿科学目标的研究,包括超
新星中微子、太阳中微子、大气中微子和地球中微子的观测以及核子衰变等方面。
\begin{figure}[!htb]
  \centering
   \includegraphics[width=\MyFactor\textwidth]{Img/chap1/mhh0}
    \caption{ 中微子的质量序列问题 }
  \label{fig:mhh0}
\end{figure}
江门中微子实验在地下七百米的岩石覆盖处建造地下实验大厅,
主要有靶质量约两万吨的液体闪烁体中心探测器,反符合水契仑柯夫探测器和顶部径迹探测器。
中心探测器需要两万吨基于线性烷苯的低本底、高透明度的液闪,
以及大约~17000~个20英寸的高量子效率光电倍增管。
中心探测器的能量分辨要求达到前所未有的3\% 水平($\sigma$/E=3\%/E)
\citep{zhan2008determination,zhan2009experimental,li2013unambiguous}。
实验关键技术挑战包括高量子效率光电倍增管和低本底液闪研制。
通过测量反应堆中微子振荡能谱来分辨中微子的质量顺序。振荡可以用下面的公式表示:
\begin{eqnarray}\label{eq:eesur}
P_{ee}(L/E)=1-P_{12}-P_{31}-P_{32}\\
P_{21} = cos^4(\theta_{13})sin^2(2\theta_{12})sin^2(\Delta_{12}) \nonumber \\
P_{31} = cos^2(\theta_{12})sin^2(2\theta_{13})sin^2(\Delta_{13}) \nonumber \\
P_{32} = sin^2(\theta_{12})sin^2(2\theta_{13})sin^2(\Delta_{32}) \nonumber
\end{eqnarray}
其中$\Delta_{ij}=\Delta m^2_{ij}*L/(4E)$,并且质量平方差之间满足$\Delta m^2_{31}=\Delta m^2_{32}
+ \Delta m^2_{21}$的关系。由于正序的中微子质量顺序满足$|\Delta m^2_{31}| > |\Delta m^2_{32}|$的关系,
而逆序的质量顺序满足$|\Delta m^2_{31}| < |\Delta m^2_{32}|$的关系。因此中微子质量顺序在反应堆
中微子能谱中存在本征的差别,此差别代表反应堆中微子的测量对中微子质量顺序的分辨能
力。因为$|\Delta m^2_{31}|\approx 30|\Delta m^2_{21}|$,
探测器需要非常好的能量分辨能力才能够区分$|\Delta m^2_{32}|,|\Delta m^2_{32}|$之间微小的差别,
参见图~\ref{fig:mhh1}~。
\begin{figure}[!htb]
  \centering
   \includegraphics[width=\MyFactor\textwidth]{Img/chap1/mhh1}
    \caption{ 正反质量序列的中微子能谱 }
  \label{fig:mhh1}
\end{figure}
从工程及物理等方面综合评估,2015年7月份中心探测器方案选定为有机玻璃罐加不锈钢支架方案~\ref{fig:junocd}~。
中心探测器液闪使用~LAB + PPO + bis-MSB~配方,
通过使用优异的原材料以及在线纯化等手段提高液闪衰减长度。
实验计划在2020年灌装及开始取数。
\begin{figure}[!htbp]
  \centering
   \includegraphics[width=\MyFactor\textwidth]{Img/chap1/junocd}
    \caption{ 江门中微子实验中心探测器 }
  \label{fig:junocd}
\end{figure}
\section{论文的内容和结构}
第一章为引言部分,简单介绍了中微子物理的基本知识,包括中微子的发现历史以及中微子振荡。
并以典型实验为例介绍了各种中微子探测方法。
后面四章为论文的主体部位,介绍了论文工作的四个主要方面。

第二章介绍~PMT~相关工作。包括大亚湾~PMT~增益刻度及时间刻度检查;
基于~Flash-ADC~的~PMT~后脉冲测试系统的搭建及数据分析工作。

%第三章主要为液闪探测器中电子学部分的相关模拟工作。首先介绍了大亚湾电子学的非线性测试。然后介绍了PMT波形重建算法的开发。
%第三部分为根据中心探测器中基本物理量的分布,为探测器电子学读出系统的设计提供参考依据。
%最后根据模拟结果,研究了电子学因素对江门实验能量分辨率的影响。
第三章为光电倍增管采样波形重建。首先是背景介绍,介绍波形采样及
波形重建的必要性。第二部分介绍波形重建算法(~Template~Fit~)的开发。
~Template~Fit~是一种模型无关的波形重建算法,不需要事先对波形有了解,
可以通过刻度数据获得模板,从而进行波形重建。最后一部分
给出了该算法的一些性能参数。

第四章介绍液闪中的正负电子鉴别,及其在压低电子型本底上的应用。
首先介绍了液闪中正电子偶素的寿命测试工作。
然后介绍了基于光子发射时间谱的不同,正负电子鉴别的模拟工作。
正电子偶素,以及正电子湮灭产生伽马光子的能量沉积的空间分布,都会对
正电子光子发光时间谱产生迟滞作用,从而引起与电子发光时间谱的差别。
最后给出了液闪中正负电子鉴别的一个应用举例:压低宇宙线带来的$^8$He/$^9$Li本底。
通过$\chi^2$分析,
在放松~muon veto~的~IBD~样本上使用正负电子鉴别可以使质量序列(~MH~)灵敏度提高0.6。

第五章是关于液体闪烁体探测器中超新星定位的研究。
首先简要介绍了超新星物理,以及江门实验对超新星的探测。
然后重点介绍了超新星定位的两种方法,一是利用~IBD~反应
快慢信号连线通过统计方法给出超新星方向,
一是利用中微子电子弹性散射,
通过重建散射电子径迹方向给出超新星方向。
其中第二种方法为
液闪探测器中的首次研究。
最后给出了不同距离的超新星的方向定位精度。

第六章为总结与展望,一些重建工作尚有优化的空间。
