
\chapter{闪烁体探测器中微子物理简介}
\label{chap:chap2}
\section{中微子的发现}
\subsection{~Pauli~提出中微子假说}
~1914~James Chadwick~在研究贝塔衰变的能谱时,发现电子的能谱是连续的~\ref{fig:beta}~。 当时人们认为贝塔衰变是两体衰变,因此电子的能谱应该是一条单能的线。观测到的连续谱违反了两体衰变过程中的能量和动量守恒。1930年12 月,~Pauli~提出了存在一种电中性,自旋为1/2的粒子的假说。并将其称之为中子,该粒子在贝塔衰变中和电子一起释放出来,解决了能量和动量不守恒的问题。两年后,中子被发现了,但是由于其质量太大,不能成为贝塔衰变中丢失的中性粒子。1934 年,~E.Ferimi~ 从理论上解释了贝塔衰变,并且给出了贝塔衰变中电子能谱的形状,并且将~Pauli~假说的中性粒子改名为中微子。
\begin{figure}[!htbp]
  \centering
   \includegraphics[width=\MyFactor\textwidth]{Img/chap1/beta}
    \caption{观测到的镭贝塔衰变连续的电子能谱}
  \label{fig:beta}
\end{figure}
\subsection{中微子的首次探测}
世界上第一用来探测中微子的探测器是 “HerrAuge” ~\ref{fig:det_1}~。 坐落在~Hanford~反应堆旁边,是~C. Cowan~ 和~ F. Reines~主导的~Poltergeist~项目的重要组成部分。但是这个~1954~年的先驱性的探测器本底远远高于信号。此后,~C. Cowan~和~ F. Reines~在~Savannah~ 河反应堆的旁边建造了一个改进的探测器~\ref{fig:det_2}~,有力的证明了中微子的存在。该探测器中中微子的探测是利用经典的反贝塔衰变
\begin{eqnarray}\label{eq:ibd}
\bar{\nu_e} + p \rightarrow n + e^+
\end{eqnarray}
\begin{figure}[!htbp]
  \centering
  \begin{subfigure}[b]{\MySubFactor\textwidth}
    \includegraphics[width=\MyFactor\textwidth*6/4]{D:/Thesis/ucasthesis-master/Img/chap1/det1}
    \caption{世界上第一个中微子探测器}
    \label{fig:det_1}
  \end{subfigure}%
  \quad\quad\quad\quad\quad\quad%add desired spacing
  \begin{subfigure}[b]{\MySubFactor\textwidth}
    \includegraphics[width=\MyFactor\textwidth*6/4]{D:/Thesis/ucasthesis-master/Img/chap1/det2}
    \caption{~Savannah~探测器使用的探测原理:中微子的反贝塔衰变}
    \label{fig:det_2}
  \end{subfigure}
  \caption{ }
  \label{fig:det}
\end{figure}
~Savannah~ 河探测器是一个液体闪烁体探测器,中心是掺Cd的水。从反贝塔衰变中出来的正电子与电子发生湮灭,出来一对背对背的伽马,可以作为快信号。反贝塔衰变中的中子在慢化后被Cd俘获,Cd激发态释放伽马退激发,中子俘获产生的伽马作为慢信号。伽马通过康普顿散射在液闪中激发出闪烁光。闪烁光被探测器四周的光电倍增管探测到。这样通过快慢信号符合给出中微子信号。多年以后,1995 年 ~F. Reines~由于首次探测到中微子被授予诺贝尔物理学奖。
\subsubsection{~$\nu_{\mu}$~和~$\nu_{\tau}$~的发现}
$\nu_{\mu}$,第二种轻子味道的中微子,1962年由~L.M. Lederman~, ~M. Schwartz~和~ J. Steinberger~ 在~ Brookhaven Alternating Gradient Synchrotron (AGS)~发现。$\nu_{\mu}$ 中微子是由反应$\pi^+ \rightarrow \mu^+ + \nu_{\mu}$ 产生的。1988年的诺贝尔奖被授予这三个人由于他们发现了第二种中微子。2000年,$\tau$型中微子在费米实验室的~Direct Observation of Nu Tau (DONUT)~实验上被发现。在这个实验中,一束800GeV 的质子束用来轰击钨金属,产生粒子shower。一部分粒子衰变成$\tau$ 子。$\tau$ 子衰变产生$\nu_{\tau}$。

\section{中微子振荡}
中微子振荡的发现是中微子物理对粒子物理和宇宙学产生深刻影响的一个里程碑。中微子振荡要求中微子拥有不同质量的质量本征态,因此中微子质量不为零。假设质量最轻的中微子~$\nu_1$~质量为零,则质量差的测量,可以给出中微子质量下限。本节主要介绍中微子振荡的发现,振荡理论以及中微子振荡的物质效应。
\subsection{中微子振荡的发现}
19世纪60年代后期,物理学家和核化学专家开始致力于对来自太阳的中微子测量。在太阳内部发生的核聚变反应过程中,~PP~ 链和~CNO~循环,都会产生大量的电子型中微子。地球上测量的太阳中微子流强约为~$6\times 10^{10}$~ ~$cm^{-2}s^{-1}$~。 图\ref{fig:sun}显示了太阳内部产生的中微子的能谱。与反应堆相反,太阳内部的~PP~ 链和~CNO~ 循环只会产生~$\nu_e$~。第一个太阳中微子实验,由~RayDavis Jr. ~领导的~Homestake~ 实验,开创了利用放射化学方法探测太阳中微子的先河。~Homestake~ 实验利用~600~吨氯乙烯,利用化学反应~Eq.(\ref{eq:clar})~探测来自太阳的$\nu_e$。
\begin{eqnarray}\label{eq:clar}
\nu_e + ^{37}Cl \rightarrow e^- + ^{37}Ar
\end{eqnarray}
\begin{figure}[!htbp]
  \centering
   \includegraphics[width=\MyFactor\textwidth]{Img/chap1/sun}
    \caption{太阳中微子能谱。图中展示了不同能量的来自~PP~链(实线)和~CNO~循环的中微子流强(虚线)}
  \label{fig:sun}
\end{figure}
经过几个星期的测量,产生的~$^{37}Ar$~从氯乙烯中分离,~$^{37}Ar$~ 通过电子俘获成为激发态的~$^{37}Cl$~,接下来~$^{37}Cl$ ~通过俄歇电子发射退激发。Davis通过正比计数器探测电子来数中微子的个数,结果发现探测到的中微子比太阳模型预测的要少很多。其他实验也有类似结果,比如~Gallex/GNO~, ~SAGE~ 和~Kamiokande~实验。这被称为太阳中微子失踪之谜。那么,到底是太阳模型出错了呢,还是中微子在从太阳中心到达地球的过程中味道发生了改变呢?于此同时,~Super-Kamiokande~实验发现来自大气的~$\nu_{\mu}$~也比理论预期的少。大气中微子主要通过~$\pi$~介子衰变 ,以及后续的~$\mu$~介子衰变产生。这样~$\nu_{\mu}$~ 与~$\nu_{\mu_e}$~的数目比应当为2:1. 有趣的是,~Super-Kamiokande~实验发现来自地球下方的大气中微子偏离预期值最远。为了研究太阳中微子丢失之谜,加拿大~SNO~实验使用~1000~t 重水去探测各种味道的中微子。重水中可以发生如下反应:
\begin{eqnarray}\label{eq:sno}
&\nu_x +e  \rightarrow \nu_x + e^- \\
& \text{(所有味道中微子都可以发生中性流过程,只有$\nu_e$可以发生带电流过程)} \nonumber \\
&\nu_e +d \rightarrow p + p + e^- \\
& \text{(仅$\nu_e$带电流过程可以发生)} \nonumber \\
&\nu_x +d  \rightarrow p + n + \nu_x \\
& \text{(所有味道中微子都可以发生中性流过程)} \nonumber
\end{eqnarray}
~SNO~实验可以测量各种味道的中微子的总流强和电子中微子的流强。为了验证和提高中微子流强的测量精度,SNO实验先后使用了~3~种探测方案,第一期使用重水,第二期使用盐水,第三期使用$^{3}He$来俘获中性流产生的中子。SNO 实验证实了来自太阳的中微子流强是守恒的,只是从太阳到地球的过程中,中微子的味道发生了变化,图~\ref{fig:sno}~显示~SNO~实验三种中微子流强的测量结果。
\begin{figure}[!htbp]
  \centering
   \includegraphics[width=\MyFactor\textwidth]{Img/chap1/sno}
    \caption{ ~SNO~实验给出的太阳$^{9}B$中微子各种味道的中微子流强}
  \label{fig:sno}
\end{figure}
\subsection{中微子振荡的理论描述}
中微子振荡的现象起源于中微子的质量本征态和味道本征态不一致,也就是参与弱相互作用的中微子态与传播过程中的中微子态不一样。这个混合机制可以和夸克混合相类比。不过由于色禁闭,我们不能探测到自由夸克。
中微子味道本征态~$\nu_e$~,~$\nu_{\mu}$~,~$\nu_{\mu}$~,可以看做是质量本征态~$\nu_1$~,~$\nu_2$~,~$\nu_3$~的如下叠加。
\begin{eqnarray}
\left(\begin{array}{c}
\nu_e \\ \nu_{\mu} \\  \nu_{\mu}
 \end{array} \right)
 =\left(\begin{array}{ccc}
      1 & 0 & 0 \\ 0 & C_{23} & S_{23} \\ 0 & -S_{23} & C_{23}
      \end{array} \right)
 \left(\begin{array}{ccc}
    e^{i\phi_1}  &  &    \\
    &  e^{i\phi_2}  &    \\
    &               & 1    \end{array} \right)
\label{eq:linearMix1}
\end{eqnarray}
其中 U被称作是~ Pontecorvo-Maki-Nakagawa-Sakata~ 矩阵。它包含了3个混合角~$\theta_{ij}$~和1个复的狄拉克相角~$\delta_D$~。另外还有两个复的~Majorana~相角。~PMNS~矩阵通常被写成如下形式:
\begin{eqnarray} U
 =\left(\begin{array}{ccc}
      1 & 0 & 0 \\ 0 & C_{23} & S_{23} \\ 0 & -S_{23} & C_{23}
      \end{array} \right)
 \left(\begin{array}{ccc}
      C_{13} & 0 & S_{13}e^{-i\delta_D} \\ 0 & 1 & 0 \\
      -S_{13}e^{-i\delta_D} & 0 & C_{13}
      \end{array} \right)
 \left(\begin{array}{ccc}
      C_{12} & S_{12} & 0 \\ -S_{12} & C_{12} & 0 \\ 0 & 0 & 1
      \end{array} \right)
 \left(\begin{array}{ccc}
    e^{i\delta_{M1}}  & 0 &  0  \\
   0 & e^{i\delta_{M2}}  & 0   \\
    0&     0          & 1    \end{array} \right) 
\label{eq:MMatrix}
\end{eqnarray}
其中$S_{ij} = sin (\theta_{ij}) $和 $C_{ij} = cos (\theta_{ij}) $。
中微子质量本征态态的演化遵循薛定谔方程。假设时刻0产生了一个电子中微子,那么0时刻的状态方程可以写作:
\begin{eqnarray}\label{eq:eq1}
|\nu(t\rangle0)\rangle = | \nu_e\rangle =U^*_{e1}|\nu_1\rangle+U^*_{e2}|\nu_2\rangle+U^*_{e3}|\nu_3\rangle .
\end{eqnarray}
中微子在空间中的传播由动量为$P_i$和能量为$E_i$的质量本征态决定。一段时间t后,中微子的状态可以写成:
\begin{eqnarray}\label{eq:eq2}
|\nu(t\rangle0)\rangle = | \nu_e\rangle =U^*_{e1}e^{-iE_1t}|\nu_1\rangle+U^*_{e2}e^{-iE_2t}|\nu_2\rangle+U^*_{e3}e^{-iE_3t}|\nu_3\rangle\neq |\nu_e\rangle .
\end{eqnarray}
这种质量本征态的叠加并不一定是味道本征态。因此,在0时刻后的某一时刻,测量到另一种味道的中微子的几率非0。我们只能探测到中微子的味道本征态,因为中微子没有电荷不能参与电磁相互作用,没有色荷,不能参与强相互作用,只能参与弱相互作用。因为质量本征态可以看做是味道本征态的叠加,因此我们可以把公式~\ref{eq:eq2}~写成如下形式
\begin{eqnarray}\label{eq:eq3}
 | \nu_{\alpha}(t)\rangle &=&\sum_k U^*_{\alpha k}e^{-iE_k t}|\nu_k\rangle \\
 &=& \sum_{\beta=e,\mu,\tau} \left ( \sum_k U^*_{\alpha k}e^{-iE_k t}U^*_{\beta k} \right )| \nu_{\beta}\rangle\\
| \nu_{k}\rangle &=& \sum_{\beta =e ,\mu,\tau}U_{\beta k}|\nu_{\beta} \rangle
\end{eqnarray}
探测到~$\nu_{\beta}$~态的几率可以由~$|\nu_{\alpha} (t) \rangle $~态在味道本征态~$\nu_{\beta}$~上的投影得到。表示如下
\begin{eqnarray}\label{eq:eq4}
P(\nu_{\alpha\rightarrow \beta }(t))={\langle \nu_{\beta}| \nu_{\alpha}(t)\rangle}^2={A_{\nu_{\alpha} \rightarrow \nu_{\beta}}(t)}^2={\sum_k U^*_{\alpha k}e^{-iE_k t}U_{\beta k}}^2 \\
=\sum_{k,j}U^*_{\alpha k}U_{\beta k}U_{\alpha j}U^*_{\beta j}e^{-i(E_k-E_j)t}
\end{eqnarray}
使用极端相对论近似,$p_k = p = E$,公式~\ref{eq:eq4}~可以写成如下形式:
\begin{eqnarray}\label{eq:eq5}
P(\nu_{\alpha\rightarrow \beta }(L/E))=
=\sum_{k,j}U^*_{\alpha k}U_{\beta k}U_{\alpha j}U^*_{\beta j}e^{-i\frac{\Delta{m_{ij}^2}L}{2E}}
\end{eqnarray}
其中$\Delta_{m_{ik}^2}=m_i^2-m_j^2$对应质量平方差。L表示探测器和中微子源的距离,E表示中微子的能量。从公式~\ref{eq:eq5}~可以看出,振荡几率是由~PMNS~混合参数和重量平方差决定的。由于混合角~$\theta_{13}$~比较小,通常的3代中微子混合可以简化成2种味道的中微子的混合~\ref{eq:eq6}~。
\begin{eqnarray}\label{eq:eq6}
P_{\nu_e  \rightarrow \nu_{\mu} } (L/E)=sin^2(2\theta)sin^2\left(\frac{\Delta m^2L}{2E}\right)
\end{eqnarray}
从公式~\ref{eq:eq6}~中我们可以看到中微子振荡的振幅由$sin^2(2\theta)$决定,中微子振荡的频率由$\Delta m^2 $决定。这是实验中经常用到的一个中微子产生几率公式。
\section{中微子探测}
\subsection{中微子与物质相互作用}
中微子没有电荷,不参与电磁相互作用;中微子没有色荷,不参与强相互作用;中微子只参与弱相互作用。
实验中观测中微子,通常有以下几种相互作用:
中微子与粒子交换~$W^{\pm}$~的
相互作用为带电流(CC)相互作用~\ref{fig:cc}~,交换~$Z^0$~的相互作用为中性流(NC)相互作用~\ref{fig:nc}~,各种味道的中微子与核子的中性流相互作用过程截面相同。弹性散射可以是中性流过程,如~$\nu_e+n \rightarrow \nu_e+n $ ~\ref{fig:es}~左;也可以是带电流过程,如~$\nu+l \rightarrow
\nu+l$ ~\ref{fig:es}~右。仅有~$\nu_e$~可以通过~NC~和~CC~过程在电子上发生弹性散射,因此在中微子电子弹性散射截面中,~$\nu_e$~截面最大。
\begin{figure}[!htbp]
  \centering
   \includegraphics[width=\MyFactor\textwidth]{Img/chap1/cc}
    \caption{ 中微子与物质相互作用的带电流过程}
  \label{fig:cc}
\end{figure}
\begin{figure}[!htbp]
  \centering
   \includegraphics[width=\MyFactor\textwidth]{Img/chap1/nc}
    \caption{ 中微子与物质相互作用的中性流过程}
  \label{fig:nc}
\end{figure}
\begin{figure}[!htbp]
  \centering
   \includegraphics[width=\MyFactor\textwidth]{Img/chap1/es}
    \caption{ 中微子与物质相互作用的弹性散射过程}
  \label{fig:es}
\end{figure}

\subsection{中微子探测方法}
\subsubsection{水切伦科夫探测器}
探测中微子能量阈值通常大于~5~MeV~。
\subsubsection{基于放射化学方法的中微子探测器}
使用$C_2 Cl_4$的实验,中微子能量阈值为~814~KeV~。使用~$Ga$~的实验,中微子能量阈值为~233~KeV~。这类实验有~Homestake~, ~Gallium Detectors~
(Gallex/GNO, SAGE) 等。最早的 ~Homestake experiment~基于~\ref{eq:clar}~原理。
使用~Gd~的实验基于反应~\ref{eq:ga}~, 最早是由~Kuzmin~在~1965~年提出的。
\begin{eqnarray}\label{ga}
\nu_e + ^{71}Ga \rightarrow ^{71}Ge + e^-
\end{eqnarray}
\subsubsection{液体闪烁体中微子探测器}
液体闪烁体探测器探测~$\bar{\nu_e}$~通常使用反贝塔衰变,能量阈值是~1.8~MeV~,这是一个带电流过程,截面大概在~$10^{-42}cm^2$~量级。
探测~$\nu_e$~通常使用$\nu_x +e ^- \rightarrow \nu_x +e ^-$。这是一个弹性散射过程,能量阈值是~200~KeV~,对所有味道的中微子均敏感。但电子类型的中微子截面要高出其他类型中微子6个数量级。液体闪烁体探测器实验的研究对象涵盖太阳中微子,可以检验标准太阳模型,反应堆中微子,地球中微子,其中84\%的热量来自~U~Th~天然衰变链。
液体闪烁体探测器对低能、射程短的粒子具有比较高的探测效率。

粒子的能量绝大部分先传递给溶剂,引起溶剂分子的电离和激发。大部分受激发分子(约90%)不参与闪烁过程,以热能的形式失去能量;其中部分激发的溶剂分子处于高能态,当其迅速地退激时,便将能量传递给周围的闪烁剂分子[第一闪烁剂 (primary scintillator)),使之受激发。受激发的高能态闪烁剂分子退激复原时,能量发生转移,在瞬间发射出光子。
闪烁液中溶剂分子占99%以上,闪烁剂分子的浓度一般在1%以下。由于各种第一闪烁剂分子固有的发光光谱各不相同,
液闪中的发光物质被称为闪烁剂(scintillator),闪烁剂通过吸收退激溶剂分子的能量,受激发发射出光子。对于大亚湾液闪来说,闪烁剂为PPO,其发射光谱值为376nm,能很好的溶解于LAB.有时为了与光电倍增管的光电阴极响应光谱相匹配,通常需加入第二闪烁剂(secondary scintillator),使之与所用的光电倍增管的光阴极波长相匹配。对于大亚湾实验来说,使用的第二闪烁剂或者成为波长移位剂为Bis-MSB,其发射光谱
峰值416nm
体闪烁过程中能量转换的任何一步,其效率都不是100%的。处于激发态的溶剂分
子(S*)或其二聚体(SS*)、闪烁剂分子(F*)或其二聚体(FF*),恢复到基态时,总有一部分能量以非光子形式(以热为主)损失掉,已经形成的光子也可被闪烁液内某些成分吸收,不能达到光电倍增管。上述各种方式的能量损失,统称为淬灭(quenching)。简言之,液体闪烁计数时由于样品杯中放射性能量传递的损失,导致样品计数效率下降的过程称为淬灭。 比如 溶剂分子、闪烁剂分子浓度过高时容易形成二聚体,称浓度淬灭。溶质本身浓度太高会引起自吸收
图3-1液体闪烁过程的机制示意图
(S*:激发的溶剂分子;F*:激发的闪烁剂分子;A: 外分子;
B:光电子;Q:热能;h:荧光分子;     能量转换)
\section{论文的内容和结构}