
\chapter{液闪中的正电子鉴别}
\label{chap:chap5}
\section{正电子及正电子偶素}
1931年9月,英国物理学家~Paul Dirac~发表了一篇论文,从数学上解释了电子反粒子的存在,与电子具有相同质量,带相反电荷。1932 年,加州理工博士后Carl D. Anderson首次利用云室证实了正电子的存在,并因此获得了1936 年的诺贝尔物理学奖。1951 年,~Martin Deutsch~ 首次在气体中观测正电子偶素,并因此获得1956 年的诺贝尔奖。图 \ref{fig:p51}
\begin{figure}[!htbp]
  \centering
  \includegraphics[width=\MyFactor\textwidth]{D:/Thesis/ucasthesis-master/Img/chap5/cloud_chamber.jpg}
  \caption{正电子发现的云室照片:粒子从云室下方进入云室,打到中间的铅板上,可以看到由于能量损失,铅板上方径迹弯曲程度较下方大。}
  \label{fig:p51}
\end{figure}
%Anderson's cloud chamber picture of cosmic radiation from 1932 showing for the first time the existence of the anti-electron. The particle enters from the bottom, strikes the lead plate in the middle and loses energy as can be seen from the greater curvature of the upper part of the track.

正负电子性质对比如下表:
\begin{table}[htbp]
\centering  % 表居中
\begin{tabular}{lcc}  % {lccc} 表示各列元素对齐方式,left-l,right-r,center-c
\hline
&电子&正电子 \\ \hline  % \hline 在此行下面画一横线
\\ 静止质量&$9.1\times10^{-31}$kg& $9.1\times10^{-31}$kg       % \\ 表示重新开始一行
\\  电荷&$-1.6\times10^{-9}$C& $1.6\times10^{-9}$C    % & 表示列的分隔线
\\自旋&$1/2\hbar$&$1/2\hbar$
\\磁矩&$-\mu_{e}$&$\mu_{e}$
\\ \hline
\end{tabular}
\caption{正负电子基本参数}
\end{table}

正电子进入物质后,会通过电离碰撞,电子.空穴产生,声子散射等过程慢化,热化。在正电子热化过程中,有一定的几率从环境中捕获一个电子,由电磁相互作用形成的亚稳定束缚态,即电子偶素。
正电子偶素为正电子和电子结合的一种亚稳态,是一种奇异原子。电子偶素就是一个电子围绕着正电子转,电子偶素的能级公式与氢原子能级的公式类似。在计算氢原子系统的约化质量时$\mu = \frac{m_em_p}{m_e+m_p}$,由于质子的质量远大于电子的质量,因此其约化质量就约等于电子的质量。对于电子偶素而言,由于正电子质量和电子相等,因此它们的约化质量为电子质量的一半,所以电子偶素的里德伯常量等于氢原子系统里德伯常量除以二,为6.8eV。由能级公式可知, 正电子偶素的束缚能是6.8eV,其形成能量阈值为6.8eV。
基态偶素的自旋由组成偶素的电子和正电子自旋的相对取向决定. 若电子自旋与正电子自旋平行,则为总自旋为1 的自旋三重态 , 记为~o-Ps~。另一种,电子正电子自旋反平行,则为自旋单态,总自旋为0,记为~p-Ps~。~p-Ps~ 在真空中的寿命为125ps\citep{karshenboim2004precision}。~p-Ps~ 主要衰变为一对能量为511KeV 的~gamma~。p-Ps 可以衰变成任意偶数个光子,但是几率随着光子数增多急剧减小,比如衰变到四光子道的几率为$1.439\times10^{-6}$。o-Ps 在真空中的寿命为$142.05\pm0.02$ ns \citep{badertscher2007improved}。 但是,实验中发现O—Ps的寿命比本征寿命要短许多,甚至缩短到几个纳秒的量级,这主要是外界磁场或者淬灭效应引起的。真空中的主要衰变模式为衰变为奇数个光子,并且随着光子个数的增多几率急剧下降,比如衰变为五光子的几率约为$1.\times10^{-6}$。

 \section{液闪中正电子偶素寿命的测量}
 \subsection{实验中使用的正电子源}
$^{22}Na$是实验中最常用的正电子源,原因是它的半衰期较长(2.6年),可以满足长期实验的要求。通常实验室中使用的源强为几pCi ~几十pCi。$^{22}Na$贝塔衰变产生的正电子能谱为连续谱,终点处最大能量为0.545MeV,平均能量为178KeV。本次实验中使用的正电子源的制作方法是,将液态NaCl源滴到很薄的作为衬底的聚酰亚胺(kapton)膜上,等待源变干,然后使用铝框固定。正电子在kapton上膜里不会形成三重态,因此不会干扰长成分寿命的测量。制作好的正电子源及$^{22}Na$ 衰变纲图如图\ref{fig:p52}
\begin{figure}[!htbp]
  \centering
  \begin{subfigure}[b]{\MySubFactor\textwidth}
    \includegraphics[width=\textwidth]{D:/Thesis/ucasthesis-master/Img/chap5/source.png}
    \caption{}
    \label{fig:p52_1}
  \end{subfigure}%
  \quad\quad\quad\quad\quad\quad%add desired spacing
  \begin{subfigure}[b]{\MySubFactor\textwidth}
    \includegraphics[width=\textwidth*6/5]{D:/Thesis/ucasthesis-master/Img/chap5/na22decay.png}
    \caption{}
    \label{fig:p52_2}
  \end{subfigure}
  \caption{(a)制作好的正电子源 (b)$^{22}Na$衰变纲图}
  \label{fig:p52}
\end{figure}
\subsection{正电子源的包装方案}
常规正电子寿命实验测量使用半衰期2.6年的$^{22}Na$正电子源,Na22贝塔正衰变成为~$^{22}Ne$~。~$^{22}Ne$~第一激发态退激发放出一个1.275MeV的伽马射线,其寿命为3.7 皮秒。将探测到1.275 MeV 的伽马作为时间起点,以及正电子的产生时间,以探测到511KeV 的湮灭伽马作为时间终点。两者的时间差,即是正电子的寿命。通常测试样品为晶体或粉末,预处理样品时,需要将源做成两片薄片,粉末样品可以通过使用磨具压片制得。测试时,将放射源放在两片样品中间,这样从源两个侧面发射出来的正电子可以全部在待测样品中湮没。为正电子寿命谱积累足够多的统计量,通常统计量要求大于$10^{6}$个,然后通过解谱,就可以获得正电子在该材料中的寿命等信息。
由于本次待测样品为油状液体,正电子源未做密封,不能直接投到液闪里,需要设计新的源包装方案。正电子源的包装方案考虑到的有三种:
\begin{enumerate}
\item 参考文献\citep{franco2011positronium} 的做法,直接将源浸在液闪里。这种方法简单易行,但正电子会在液闪中扩散,正电子不再是点源,此时两个探头测到的时间差就有了来自事例顶点的偏移,会应严重影响正电子的寿命谱。
\item 直接做一个新的源,kapton膜外面镀一薄层(比如 25 微米)聚乙烯或特氟龙密封保护。这两种材料均具有非常好的防水功能,化学性质稳定,可以保护在液闪中的正电子源。但由于此源不是豁免源,制作有困难。
\item 使用3D打印机打印支撑,使源紧贴液闪放置。但这样至少一半正电子没进入液闪,除被测样品外,打印模具材质会干扰测试结果,因此不被采用。
\end{enumerate}
最终使用的方案为一次性有机玻璃比色皿结合聚酰亚胺膜的方法。一次性比色皿为从Sigma-Aldrich公司订购德国产普兰德~PMMA~ 材质比色皿,规格为~10mm*10mm*45mm~。 采用注塑模具生产,无缝。与液闪兼容性好体积小,对探头的所张立体角大,计数率高。正电子源置于两个比色皿之间,上比色皿用胶带封上,且因为倒置可不装满液闪,胶带易于贴上,下比色皿装满液闪,直接覆盖一层无胶的kapton膜,使用胶带覆盖,操作复杂,贴合过程中易出现气泡。此时正电子源再置于两个比色皿之间,正电子在湮灭前,经过的介质只有~kapton~膜和液闪,并且~kapton~胶带及~kapton~膜的影响,可通过标准样品的测试扣到源里。通过以上方法成功解决了正电子在液体中寿命的测试问题。最终的源的包装方案见图\ref{fig:p53}
\begin{figure}[!htbp]
  \centering
  \includegraphics[width=\MyFactor\textwidth]{chap5/yuanfangan}
  \caption{正电子源的包装:上下为规格10ml的一次性~PMMA~比色皿,封口材料为7.5微米厚的聚酰亚胺,源片置于中间进行测量}
  \label{fig:p53}
\end{figure}
\subsection{正电子淹没寿命谱仪与测量}
正电子在液体闪烁体中的寿命的测量,在高能所核分析室开展。使用的寿命谱仪是自主设计的快、慢符合谱仪,图\ref{fig:p54} 展示了其电子学原理。正电子寿命谱仪探头使用的~$BaF_{2}$~闪烁体晶体,两个~$BaF_{2}$~晶体需要对称地放置在两片待测样品表面。~$BaF_{2}$~闪烁体后端接上光电倍增管,探测~gamma~ 光子。%读出电子学系统的触发,使用双重符合,可以极为有效的降低本底。一重符合为1.27MeV 和511KeV 的伽马的能量符合。另一重为BaF晶体自身快慢信号的符合。该系统的时间分辨率为210皮秒。
~$BaF_{2}$~端面及窗口面附上黑色的胶带,可以有效阻止杂散光打到PMT光阴极上。恒比微分甄别器(~CFDD~583)可以同时探测伽马光子的能量以及定时,分别挑选~1.27MeV~ 和~511KeV~伽马作为起始信号和终止信号,经过延时后,时间差被时幅转换器(~TAC~) 转换成一个脉冲信号送入多道分析器(~MCA~),根据脉冲幅度记录到相应的道上(代表不同的时间)。积累足够多的统计量,即可获得寿命谱。
\begin{figure}[!htbp]
  \centering
  \includegraphics[width=\MyFactor\textwidth*4/3]{chap5/daq}
  \caption{寿命谱仪快-慢符合电子学系统}
  \label{fig:p54}
\end{figure}
谱仪的时间刻度由~$^{60}Co$~标定,图\ref{fig:p54}展示了~$^{60}Co$~的衰变纲图和伽马能谱,我们可以看到~$^{60}Co$~可以衰变出两个能量分别是1.173和1.333MeV 的伽马射线。可以认为是同时发出的。为了使得利用~$^{60}Co$~测得的时间分辨率接近真实测试时使用~$^{22}Na$~的时间分辨。使用1.333MeV的信号替代~$^{22}Na$~1.275MeV 的信号作为计时开始信号。使用1.17MeV经过康普顿散射,角度挑选后的伽马代替~$^{22}Na$~511KeV 的终止信号。经过刻度后,系统的时间分辨用半高宽度(FWHM)表示为190ps。
\begin{figure}[!htbp]
  \centering
   \includegraphics[width=\MyFactor\textwidth]{chap5/co60}
  \includegraphics[width=\MyFactor\textwidth]{chap5/gammarayco60}

  \caption{~$^{60}Co$~衰变纲图和伽马能谱}
  \label{fig:p54}
\end{figure}

实验过程中的需要排掉液闪中的氧气。由于氧气呈现出电负性,会缩短正电子的寿命。本次实验承装液体闪烁体的容器为4.5ml的比色皿,通常的排气装置不适用。此处的做法为在连接减压阀的软管的尾端加上一段一次性注射器,通过注射器的细小的针头往液闪中通氮气。通气速度由减压阀控制,4.5ml样品通气时间为40min左右。
在正电子湮没寿命测试实验中,对寿命谱的解谱工作是非常重要的。对正电子淹没寿命谱的分析,首先需要根据实验建立一个合理的模型,模型需要考虑寿命谱中分量成分的总个数,各分量成分的贡献可以用指数来描述。另外由于我们的系统还存在时间分辨率,并不是冲击响应,所以需要用指数成分卷积上一个用高斯函数描述的时间分辨。同时,依据标准样品评估出来的正电子基底材料在寿命谱中的贡献份额,也需要作为初始参数,参与到待测样品寿命谱的拟合当中。
一个流行的寿命谱解谱软件是基于最小二乘法的~Life Time~9.0 (LT9) 程序\citep{kansy1996microcomputer}。本人另基于RooFit 写出一种卷积了系统时间分辨的解谱方法,该方法提供了解谱结果的~cross check~。并且使用这两种方法的差别,作为解谱结果的误差。
测试之前,先测标准样品(Ni) ,已知标准样品中正电子湮灭寿命为110 ps\citep{zhang2015accurate}。~kapton~ 膜中,正电子寿命386$\pm$7 ps\citep{plotkowski1988positron}。根据标样测试结果估计源包装材料的影响。源包装材料的影响见图\ref{fig:p55}。通过对标准样品测试结果的解谱,得到正电子在基底材料(包括~kapton~膜以及源和样品反射面)中的湮灭几率约为$33\%$。解谱得到的结果正电子在~kapton~膜中的寿命为 373$\pm$12 ps,比例为$31\%$. 另外源中还存在占总强度约为$2\%$的长成分杂质。
后续的测试的样品包括LAB,LAB 加上不同浓度的PPO,另外考虑到实验的长期运行,液闪会出现老化(比如液闪衰减长度减小等),还测试了正电子在老化液闪中的寿命。
\begin{figure}[!htbp]
  \centering
   \includegraphics[width=\MyFactor\textwidth*6/4]{Img/chap5/sourceimpact}
    \caption{标样测试结果:通过标准样品测试评估源包装材料的影响}
  \label{fig:p55}
\end{figure}
~LAB~测试分析结果如图\ref{fig:labs}所示.图片下方画的是~fit-data~除以~data~个数的分布。解谱假设~LAB~中有两种成分,解谱结果表明这两成分的比例为:43.7\%,55.3\%。其中~44.7\%~对应长寿命成分,寿命为3.10$\pm$0.07 ns。另有一短寿命成分,寿命值为0.155$\pm$0.085 ns。从残差图可以看出数据点被很好地控制在~2$\sigma$~范围内。并且RooFit 给出的~reduced chi square~也偏离1不远,为1.15。
\begin{figure}[!htbp]
  \centering
   \includegraphics[width=\MyFactor\textwidth*6/4]{Img/chap5/labr}
    \caption{正电子在LAB中的寿命谱的解谱}
  \label{fig:labs}
\end{figure}
老化液闪的测试中,液闪老化的方法是将液闪装在304钢罐中,保持其70摄氏度的高温,并且持续时间是6个月。图\ref{fig:oldrf} 展示了老化液闪和未经老化的液闪的寿命谱的对比。通过对比,我们可以看到老化液闪长寿命成分相比普通液闪有略微降低。解谱结果显示老化液闪的长寿命成分降低为 2.860$\pm$0.074 ns。 对于参杂了不同浓度PPO的液闪的测量结果如\ref{tab:ppo}所示。从解谱结果可以看出,~p-Ps~的形成几率以及寿命值对与PPO浓度不是特别敏感。
\begin{figure}[!htbp]
  \centering
   \includegraphics[width=\MyFactor\textwidth*6/4]{Img/chap5/oldr}
    \caption{正电子在经过半年时间老化的LAB中的寿命谱与正常液闪的对比}
  \label{fig:oldrf}
\end{figure}

\begin{table}[htbp]
\centering  % 表居中
\begin{tabular}{lcc}  % {lccc} 表示各列元素对齐方式,left-l,right-r,center-c
\hline
&强度[\%]&寿命[ns] \\ \hline  % \hline 在此行下面画一横线
\\ 1 g/L PPO &40.0(1.1)  & 3.194(0.043) % \\ 表示重新开始一行
\\ & 60.0(1.1) &0.1590(0.0039)
\\ 10g/L PPO&41.06(0.82)               & 3.128(0.032)                            % & 表示列的分隔线
\\  &58.94(0.82)               &0.1603(0.0038)
\\  30g/L PPO&46.4(1.3)                 &3.100(0.030)
\\  & 53.6(1.3)                &53.6(1.3)
\\ \hline
\end{tabular}
\caption{正电子在参杂不同PPO浓度的LAB中的寿命谱解谱结果}
\label{tab:ppo}
\end{table}

\section{正负电子鉴别能力的模拟工作}
\subsection{模拟及鉴别量的构造}
大亚湾使用的模拟软件为~NuWa-trunk~,在其软件包~MDC09b~中,正电子偶素的形成过程没有加入到正电子与液闪的相互作用过程当中。江门中微子实验的离线模拟软件~SNiper~中,正电子的物理过程列表中也没有正电子偶素的形成这一过程。这两套模拟软件使用的是~Geant4.9.4~。 实际上,正电子偶素的产生过程在~Geant4.0.2~中已经被实现,但是并不在普通的低能电磁相互作用类中,而是在一个叫做~DNA~ 的电磁相互作用过程当中。具体可参见网址\href{www.geant4.org/geant4/support/ReleaseNotes4.9.2.h}{Geant4}。%
为了简单起见,我们直接给由于正负电子湮灭产生的光学光子加上了一个~tag~。然后通过抽样~p-Ps~形成几率及寿命,决定一个正电子是否形成了长寿命偶素以及偶素的寿命,然后将这个寿命延迟量加到总的~hit time~里。图\ref{fig:p56}展示了经过平均后的正电子和电子的光子发射时间谱的分布。所有正电子电子事例均在探测器中心,零时刻产生。从图中可以看出正电子事例中,晚到达~PMT~ 的光学光子所占比例相比电子有提高。
\begin{figure}[!htbp]
  \centering
   \includegraphics[width=\MyFactor\textwidth]{Img/chap5/epluspetd}
    \caption{多个事例平均后的电子和正电子的光子发射时间谱的分布,所有正电子和电子事例均在探测器中心零时刻产生,扣除飞行时间后的~hit time~分布即可认为是光子发射时间分布}
  \label{fig:p56}
\end{figure}
为了通过光子发射时间的分布~PETD~来鉴别电子、正电子,需要构造鉴别量。其中有两个手动构造的鉴别量,参考文献\citep{gatti1961new}的方法,分别叫做~T2T~ 鉴别量和~Gatti Parameter~鉴别量。~T2T~ 鉴别量是~Tail To Total Ratio~的简称。通过光子发射时间谱尾部所占比例鉴别正负电子。通过优化,发现~tail~ 从10 ~ns~ 后计算开始,并且整个光子发射时间窗口选作50~ns~,可以取得比较好的鉴别效果。~Gatti Parameter~的定义如方程 Eq. ( \ref{eq:gatti} ) 所示。其中的1 和2代表我们需要鉴别的两种粒子。~$P_{12}$~代表~PETD~ 上光子在该时段发射的几率。
\begin{equation}\label{eq:gatti}
G=\sum_{n}{r_{i}(t_{n}w(t_{n}))}
\end{equation}

 \begin{equation*}
 r_{i}(t_{n})=\int_{t_{i-1}}^{t_{i}}P_{12}(t)dx
 \end{equation*}
 \begin{equation*}
 w(t_{n})=\frac{r_{1}(t_{n})-r_{2}(t_{n})}{r_{1}(t_{n})+r_{2}(t_{n})}
 \end{equation*}
除了上述两种鉴别量外,还尝试了~ROOT~中提供的人工神经元网络方法。数据训练基于多层感知器算法,输入层有16个节点,分别是总5\%,7\%,9\%等特征时刻点。中间加入两层隐藏层,通过训练分别分配给16个节点合适的权重。输出层为一个表示粒子种类的鉴别量。~TTree~中每隔一个事例分别作为训练和测试样本。如果训练样本的误差远小于测试样本,则表示有过度训练。图\ref{fig:p57} 是这一方法的示意图,其中连接个节点的线的粗细表征了其权重。
\begin{figure}[!htbp]
  \centering
   \includegraphics[width=\MyFactor\textwidth*6/4]{Img/chap5/mlp}
    \caption{人工神经网络方法鉴别正负电子:输入层有16个表征光子发射时间谱特征的时刻点量,输出为一个表征粒子种类的鉴别量}
  \label{fig:p57}
\end{figure}
为了评估粒子鉴别的效果,构造了一个叫做~FOM~$(~figure of merit~)$的变量,定义为正负电子~psd~ 变量分布图中心值的差值除以~psd~变量分布图~sigma~值的和。分别在同一样本上使用最优的~Gatti parameter~鉴别方法和多变量分析中的人工神经元网络方法,其结果如图\ref{fig:p58}。按照上述~FOM~的定义,使用~Gatti parameter~ 鉴别方法最终的FOM值为0.611,使用人工神经元网络方法的~FOM~值为0.612。 两者的鉴别能力接近。后续研究使用~Gatti parameter~鉴别方法。
\begin{figure}[!htbp]
  \centering
   \includegraphics[width=\MyFactor\textwidth*7/4]{Img/chap5/fomcmp}
    \caption{手动构造~Gatti parameter~方法和神经网络方法正负电子鉴别效果对比}
  \label{fig:p58}
\end{figure}
我们使用江门模拟软件研究不同能量下正负电子的鉴别能力。其中的关键参数能标大约为~$1200/MeV$~。PMT 光阴极覆盖率大概为~78\%~。 在模拟中,加入了1ns ~PMT~的渡越时间涨落,这会给光子发射时间谱带来一定的~smear~。 为了简单方便的描述最终的鉴别效果,我们把每个能量下的~Gatti parameter~的分布近似成一个高斯分布,用中心值和~sigma~来代替真实的~psd~鉴别量的分布。不同能量下,正负电子~psd~鉴别量的分布特征如图\ref{fig:p59} 所示。其他能量的~psd~鉴别量的分布通过三次样条差值获得。图\ref{fig:p510} 为正负电子叠放在一起的效果。从图中我们可以看到,对于高能正负电子,鉴别能力非常小。

\begin{figure}[!htbp]
  \centering
  \begin{subfigure}[b]{\MySubFactor\textwidth}
    \includegraphics[width=\textwidth]{D:/Thesis/ucasthesis-master/Img/chap5/m1}
    \caption{}
    \label{fig:p59_1}
  \end{subfigure}%
  \quad\quad\quad\quad\quad\quad%add desired spacing
  \begin{subfigure}[b]{\MySubFactor\textwidth}
    \includegraphics[width=\textwidth]{D:/Thesis/ucasthesis-master/Img/chap5/m2}
    \caption{}
    \label{fig:p59_2}
  \end{subfigure}
   \begin{subfigure}[b]{\MySubFactor\textwidth}
    \includegraphics[width=\textwidth]{D:/Thesis/ucasthesis-master/Img/chap5/s1}
    \caption{}
    \label{fig:p59_3}
  \end{subfigure}%
  \quad\quad\quad\quad\quad\quad%add desired spacing
  \begin{subfigure}[b]{\MySubFactor\textwidth}
    \includegraphics[width=\textwidth]{D:/Thesis/ucasthesis-master/Img/chap5/s2}
    \caption{}
    \label{fig:p59_4}
  \end{subfigure}
  \caption{(a)电子~psd~变量分布均值 (b)正电子~psd~变量分布均值 (c)电子~psd~变量分布~sigma~ (b) 正电子~psd~ 变量分布~sigma~}
  \label{fig:p59}
\end{figure}



\begin{figure}[!htbp]
  \centering
   \includegraphics[width=\MyFactor\textwidth*5/3]{Img/chap5/compsm}
    \caption{不同能量下~psd~变量的分布特征}
  \label{fig:p510}
\end{figure}
一个比较有意思的题目是假设我们的电子学不能够区分多光子的情况,即是如果有多个光子以比较短的时间间隔打到~PMT~ 上,最终数据中我们不能区分开这些光子,那么还能根据正负电子光子发射时间谱来鉴别正负电子吗?假设电子学时间读出系统具有50 ns 的死时间,会把在50~ns~内到达的~hit~合并成一个~hit~。 图\ref{fig:p511} 上方为死时间很小(2ns)的情况,下方为假设死时间50~ns~的情况。较长的死时间会淹没正负电子光子发射时间分布的差别,导致最终没有鉴别正负电子的能力。

\begin{figure}[!htbp]
  \centering
   \includegraphics[width=\MyFactor\textwidth*6/4]{Img/chap5/fee}
    \caption{电子学死时间对正负电子鉴别的影响:假设电子学时间读出系统具有50~ns~的死时间(典型的TDC插件的死时间)}
  \label{fig:p511}
\end{figure}
\subsection{大亚湾电子正电子事例对模拟的初步验证}
大亚湾的电子学前端系统存在约~60~ns 的死时间图\ref{fig:feedead},因此不能区分多光子的情况。如果一个~PMT~只有一个光子击中,则这种影响可以不用考虑。因此我们挑选低能以及在探测中心0.5m球内的事例,使得多数~PMT~只接受到一个光电子。电子型事例来自~Bi212~\ref{fig:bi212}。$^{212}Bi$贝塔衰变到$^{212}Po$,随后$^{212}Po$以0.3微妙的半衰期经过阿尔法衰变衰变成$^{208}Pb$ 。基于此特点,$^{212}Bi$事例的挑选可以通过类似IBD挑选的以下步骤实现:
\begin{enumerate}
\item 慢信号能量范围从1.1MeV到1.4MeV
\item 快慢信号时间间隔大于1微妙,小于3微妙
\item 快慢信号顶点距离小于0.7米
\end{enumerate}

正电子型的事例有IBD快信号提供。使用的数据来自~P14A~。因为需要知道光子发射时间,即是每个~PMT~的时间信息,因此只能使用全重建,处理一个~run~的~raw data~需要大概9个小时。为了此研究目的,IBD挑选条件如下:
\begin{enumerate}
\item 快慢信号距离小于1米
\item 考虑到与电子型信号的能量的匹配,沉积能量在2.2MeV附近,快信号能量范围需要在1.25至1.4MeV范围内
\end{enumerate}
上述挑选的电子正电子型事例的光子发射时间谱经过平均后的结果如图\ref{fig:feep}所示。使用~Gatti Parameter~鉴别后的结果如图\ref{fig:datar}所示。我们可以看到相比于模拟数据,真实正负电子样本的~Gatti Parameter~的分布涨落较大。这些效应来自前段电子学,以及我们的特殊的事例挑选。使用大亚湾的电子学模拟软件可以部分重现这些问题\ref{fig:inpt_1},在其鉴别量分布图上也出现一些奇怪的涨落结构。图\ref{fig:inpt_2}是Broex的政府电子的鉴别效果,作为对比,放在这里。

\begin{figure}[!htbp]
  \centering
   \includegraphics[width=\MyFactor\textwidth*6/4]{Img/chap5/feedead}
    \caption{大亚湾FEE TDC分布}
  \label{fig:feedead}
\end{figure}
\begin{figure}[!htbp]
  \centering
   \includegraphics[width=\MyFactor\textwidth*6/4]{Img/chap5/bi212}
    \caption{$^{212}Bi$ 衰变链}
  \label{fig:bi212}
\end{figure}
\begin{figure}[!htbp]
  \centering
   \includegraphics[width=\MyFactor\textwidth*6/4]{Img/chap5/feep}
    \caption{正负电子样本中的光子发射时间谱}
  \label{fig:feep}
\end{figure}
\begin{figure}[!htbp]
  \centering
   \includegraphics[width=\MyFactor\textwidth*6/4]{Img/chap5/datar}
    \caption{大亚湾正负电子样本鉴别效果}
  \label{fig:datar}
\end{figure}
\begin{figure}[!htbp]
  \centering
  \begin{subfigure}[b]{\MySubFactor\textwidth}
    \includegraphics[width=\textwidth*5/4]{D:/Thesis/ucasthesis-master/Img/chap5/inpt1}
    \caption{}
    \label{fig:inpt_1}
  \end{subfigure}%
  \quad\quad\quad\quad\quad\quad%add desired spacing
  \begin{subfigure}[b]{\MySubFactor\textwidth}
    \includegraphics[width=\textwidth*6/5]{D:/Thesis/ucasthesis-master/Img/chap5/inpt2}
    \caption{}
    \label{fig:inpt_2}
  \end{subfigure}
  \caption{(a)经过大亚湾电子学模拟的正负电子鉴别效果 (b)Broex 实验组正负电子鉴别效果图}
  \label{fig:inpt}
\end{figure}
\section{正负电子鉴别技术压低~$^{8}He/^{9}Li$~本底的研究}
\subsection{概要介绍}
江门中微子实验是一个多物理目标的实验,对谱形的精确测量要求达到了前所未有的3\%能量分辨率的水平。其中宇宙线带来的~$^{8}He/^{9}Li$~本底是非常重要的一个本底。~$^{8}He/^{9}Li$~的衰变模式如下表所示。

\begin{table}[htbp]
\centering  % 表居中
\begin{tabular}{lcc}  % {lccc} 表示各列元素对齐方式,left-l,right-r,center-c
\hline
&~$^{8}He$~&~$^{9}Li$~ \\ \hline  % \hline 在此行下面画一横线
\\ 半衰期&119.0(15) ms& $178.3(4)$ ms       % \\ 表示重新开始一行
\\ 衰变模式&$\beta^{-}$,n (50.8\%)~~~$\beta^{-}$(49.2\%) & $\beta^{-}$,n (83.1\%) , ~~~$\beta^{-}$(16.0\%)   % & 表示列的分隔线
\\ \hline
\end{tabular}
\caption{~$^{8}He/^{9}Li$~ 的半衰期与衰变模式}
\end{table}
其中贝塔中子级联衰变可以模拟一个真实的中微子信号。由于真实的中微子信号为以正电子为快信号和中子为慢信号的双重符合信号,所以有可能通过正负电子鉴别压低一部分这样的本底。不使用正负电子鉴别技术,综合考虑同位素和muon track 的关系以及活时间等因素后,目前的veto 方案如下图\ref{fig:p512} :
\begin{figure}[!htbp]
  \centering
   \includegraphics[width=\MyFactor\textwidth*3/2]{Img/chap5/vetoo}
    \caption{不适用正负电子鉴别技术,优化过后的~muon veto ~方案}
  \label{fig:p512}
\end{figure}

经过这样的~veto~方案之后,每天的~Li9~本底个数降低为1.6个.信号的个数为60个.本底个数已经非常少,使用粒子鉴别技术意义不大。
本研究的基本思想是通过改变veto方案,增加本底信号比值,然后通过~psd~技术降低本底,研究对最终的质量序列灵敏度的影响。比如我们通过将~veto~时间从原来的1.2s 降低到0.7s,~veto~使用的柱形圆柱的cut 从3m降低到1.6m,那么信号的个数会从一天60个上升到70个,相应的本底个数也从一天1.6个增加到16.6个。
新样本~psd~鉴别量的分布依据图\ref{fig:p510}产生。在这个过程中使用了一些假设和近似,分别如下:
\begin{enumerate}
\item 对与特定的能量的正电子、电子而言,~Gatti Parameter~鉴别量的分布用高斯来近似
\item 使用的中微子快信号能谱假设为反应堆能谱加上正的质量序列等振荡参数产生的,不考虑探测器的非线性,如图\ref{fig:p513}所示
\item 使用的~$^{8}He/^{9}Li$~能谱如图\ref{fig:p514}所示,与\citep{an2015neutrino}保持一致。
\item 假设偶然符合本底只受活时间的影响,快中子和alpha n 本底不随~psd~和muon veto变化,绝对数目保持不变
\end{enumerate}
\begin{figure}[!htbp]
  \centering
   \includegraphics[width=\MyFactor\textwidth]{Img/chap5/sigs}
    \caption{江门中微子实验inverse beta decay (IBD)快信号能谱}
  \label{fig:p513}
\end{figure}

\begin{figure}[!htbp]
  \centering
   \includegraphics[width=\MyFactor\textwidth]{Img/chap5/bkgs}
    \caption{江门中微子实验~$^{8}He/^{9}Li$~ 电子能谱}
  \label{fig:p514}
\end{figure}
分别从IBD快信号能谱和~$^{8}He/^{9}Li$~ 电子能谱中抽取正电子和电子能量,然后根据图\ref{fig:p510} 的差值,得到~Gatti Parameter~ 鉴别量分布的均值和方差,然后抽样出具体的~Gatti Parameter~值。填成直方图后,如图\ref{fig:p515}所示。在信号和本底重叠区域(图\ref{fig:p515}中黄色区域),优化~psd cut~,给出最优的灵敏度。
\begin{figure}[!htbp]
  \centering
   \includegraphics[width=\MyFactor\textwidth]{Img/chap5/gattisb}
    \caption{抽样得到的IBD快信号和~$^{8}He/^{9}Li$~ 电子事例的~psd~鉴别量的分布}
  \label{fig:p515}
\end{figure}
\subsection{加入~psd~后的chi suqare函数的构造}
主要参考\citep{li2013unambiguous},将观测到的能谱(含信号和本底)分成200个~bin~。考虑反应堆及振荡参数不确定性的形象,chi square 可以写成如下形式(待写):由于加入了~psd~,引入了新的误差来源,需要增加新的误差项。目前假设~psd~带来的误差是~bin2bin ~ 非关联的,考虑到~psd~在不同能量点的效率的不同,将新增误差项写成$${\left((1-\epsilon_{i})\times{10\%}\times{N_{i}}\right)}^{2}$$的形式,其中$N_{i}$ 表示该能量bin的信号或者本底的个数。psd cut 对信号和本底的效率如图\ref{fig:p516}和 \ref{fig:p517} 所示。优化psd cut的方法如图\ref{fig:p518}所示。
\begin{figure}[!htbp]
  \centering
   \includegraphics[width=\MyFactor\textwidth*3/2]{Img/chap5/effshow1}
    \caption{psd cut对信号的效率}
  \label{fig:p516}
\end{figure}
\begin{figure}[!htbp]
  \centering
   \includegraphics[width=\MyFactor\textwidth]{Img/chap5/effshow2}
    \caption{psd cut对本底的效率}
  \label{fig:p517}
\end{figure}
\begin{figure}[!htbp]
  \centering
   \includegraphics[width=\MyFactor\textwidth]{Img/chap5/cuts}
    \caption{psd cut的优化}
  \label{fig:p518}
\end{figure}
按照上述方法,我们研究不同~veto~方案下,经过~psd~后灵敏度的大小。最佳的灵敏度来自1s 和2m这组veto 方案,经过psd后,chi square 的平方可以达到11.06,相比未做psd的默认方案,提高了4.3\%. 结果的总结见图\ref{fig:p519}。
\begin{figure}[!htbp]
  \centering
   \includegraphics[width=\MyFactor\textwidth*7/4]{Img/chap5/sum}
    \caption{不同~veto~ 方案经过~psd cut~后的~chi square ~平方}
  \label{fig:p519}
\end{figure}
