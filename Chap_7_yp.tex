
\chapter{总结与展望}
\label{chap:chap7}
\section{论文总结}

本论文包括四个方面的工作内容:
%第一部分介绍大亚湾的~PMT~增益刻度工作以及在新型MCP-PMT 测试工作中承担的后脉冲测试工作。
%第二部分主要为液闪探测器中电子学部分的相关模拟工作。首先介绍了大亚湾电子学的非线性测试。然后介绍了PMT波形重建算法的开发。第三部分为根据中心探测器中的基本物理量的分布,为探测器电子学读出系统的设计提供参考依据。最后根据模拟结果,研究了电子学因素对江门实验能量分辨率的影响。
%第三部分介绍液闪中的正负电子鉴别,及其在压低电子型本底上的应用。首先介绍了液闪中正电子偶素的寿命测试。然后介绍了在模拟中添加正电子偶素对光子发光时间谱的迟滞作用。最后给出了液闪中正负电子鉴别的一个应用举例,可以用于压低宇宙线带来的~$^8$He/$^9$Li ~本底。
%第四部分是关于液体闪烁体探测器中超新星定位的研究。首先简要介绍了超新星物理,以及江门实验对超新星的探测,然后重点介绍了超新星定位的两种方法,其中第二种利用中微子电子弹性散射定位为在液闪探测器中的首次尝试。最后给出了不同距离的超新星的方向定位精度。
第一部分介绍~PMT~相关工作。包括大亚湾~PMT~增益刻度及时间刻度检查;
基于~Flash-ADC~的~PMT~后脉冲测试系统的搭建及数据分析工作。
第二部分为光电倍增管采样波形重建。首先是背景介绍,介绍波形采样及
波形重建的必要性。第二部分介绍波形重建算法(~Template~Fit~)的开发。
~Template~Fit~是一种模型无关的波形重建算法,不需要事先对波形有了解,
可以通过刻度数据获得模板,从而进行波形重建。最后一部分
给出了该算法的一些性能参数。
第三部分介绍液闪中的正负电子鉴别,及其在压低电子型本底上的应用。
首先介绍了液闪中正电子偶素的寿命测试工作。
然后介绍了基于光子发射时间谱的不同,正负电子鉴别的模拟工作。
正电子偶素,以及正电子湮灭产生伽马光子的能量沉积的空间分布,都会对
正电子光子发光时间谱产生迟滞作用,从而引起与电子发光时间谱的差别。
最后给出了液闪中正负电子鉴别的一个应用举例:压低宇宙线带来的$^8$He/$^9$Li本底。
通过$\chi^2$分析,
在放松~muon veto~的~IBD~样本上使用正负电子鉴别可以使质量序列(~MH~)灵敏度提高0.6。
第四部分是关于液体闪烁体探测器中超新星定位的研究。
首先简要介绍了超新星物理,以及江门实验对超新星的探测。
然后重点介绍了超新星定位的两种方法,一是利用~IBD~反应
快慢信号连线通过统计方法给出超新星方向,
一是利用中微子电子弹性散射,
通过重建散射电子径迹方向给出超新星方向。
其中第二种方法为
液闪探测器中的首次研究。
最后给出了不同距离的超新星的方向定位精度。

\section{展望}

目前的能量重建的输入几乎全部来自于探测器模拟。未经过电子学模拟及波形重建。经过波形刻度之后(本论文波形拟合部分),可以重建出波形对应的电荷数,问题是我们怎样由电荷数拿到对应的光电子数,一个可能的方法是使用贝叶斯方法,如公式\ref{eq:fullrec}。假设我们的电荷读出窗口为$t_1$,$t_2$,$P_N$表示在读出窗口时间窗内的hit个数的分布。时间窗内hit个数除了受到粒子能量沉积影响外,还会受到比如后脉冲,暗噪声率等的影响。$P_Q$由PMT的电荷分辨决定,例如只有一个光电子时,$P_Q$即为单光电子谱。
\begin{eqnarray}\label{eq:fullrec}
P_{N}(n|q,t_1,t_2)=\frac{P_Q(q|n)P_N(n|t_1,t_2)}{P_Q(q|t_1,t_2)}\\
=\frac{P_Q(q|n)P_N(n|t_1,t_2)}{\sum_{i=0}^{\infty}P_Q(q|i)P_N(i|t_1,t_2)}   \nonumber
\end{eqnarray}
上述PMT电荷重建方法为比较真实的重建方法,可尝试在软件框架中实现。
